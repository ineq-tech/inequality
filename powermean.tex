\documentclass{subfile}

\begin{document}
	\section{Bernoulli and Power Mean Inequality}\label{sec:powermean}
	\index{Bernoulli inequality}\emph{Bernoulli inequality} is a well known result and it has some nice consequences. We mentioned earlier that we will show more proofs of arithmetic-geometric mean inequality. We will show one such proof here using this inequality.
		\begin{theorem}[Bernoulli's inequality]
			Let $n$ be a positive integer and $x>-1$ be a real number. Then
				\begin{align*}
					(1+x)^{n}
						& \geq1+nx
				\end{align*}
		\end{theorem}
	\index{Generalized Bernoulli inequality}This inequality can be generalized as the following result.
		\begin{theorem}[Generalized Bernoulli inequality]
			Let $x_{1},\ldots,x_{n}>-1$ be real numbers such that either all are positive or all are negative. Then
				\begin{align*}
					(1+x_{1})\cdots(1+x_{n})
						& > 1+x_{1}+\ldots+x_{n}
				\end{align*}
		\end{theorem}% Do we use Hadžiivanov, Nikolaĭ; Prodanov, Ivan Bernoulli's inequalities?
	
		%\begin{definition}[Power mean or Generalized mean]
			Let $a_{1},\ldots,a_{n}$ be positive real numbers. Then the \index{Power mean}\emph{generalized mean} or \emph{power mean of order} $r$ is defined as
				\begin{align*}
					\mathfrak{M}_{r}(a_{1},\ldots,a_{n})
						& = \left(\dfrac{a_{1}^{r}+\ldots+a_{n}^{r}}{n}\right)^{\frac{1}{r}}
				\end{align*}
			Note that the arithmetic mean of $a_{1},\ldots,a_{n}$ is actually $\mathfrak{M}_{1}(a_{1},\ldots,a_{n})$. Similarly, the harmonic mean is $\mathfrak{M}_{-1}(a_{1},\ldots,a_{n})$. Moreover, $\mathfrak{M}_0(a_{1},\ldots,a_{n})$ is the geometric mean which we show below.
		%\end{definition}
	We may omit the numbers $a_{1},\ldots,a_{n}$ and just call it $\mathfrak{M}_{r}$ instead of $\mathfrak{M}_{r}(a_{1},\ldots,a_{n})$ if the context is clear. An even better way to denote this would be using $\mathfrak{M}_{r}(\mathbf{a})$ where $\mathbf{a}=(a_{1},\ldots,a_{n})$.
		\begin{align*}
			\mathfrak{M}_{r}(\mathbf{a})
				& = \dfrac{\|\mathbf{a}\|_{r}}{\sqrt[r]{n}}
		\end{align*}
	We can also denote the arithmetic, geometric and harmonic means of $\mathbf{a}$ by $\mathfrak{A}(\mathbf{a}),\mathfrak{G}(\mathbf{a})$ and $\mathfrak{H}(\mathbf{a})$ respectively. The notations $\mathfrak{M},\mathfrak{A},\mathfrak{G},\mathfrak{H}$ are inspired by \textcite{hardy_littlewood_polya_1934}. 
	
	Next, we show that $\mathfrak{G}(\mathbf{a})$ is the geometric mean.
		\begin{align*}
			\mathfrak{G}(\mathbf{a})
				& = \lim\limits_{r\to0}\left(\dfrac{a_{1}^{r}+\ldots+a_{n}^{r}}{n}\right)^{\dfrac{1}{r}}
		\end{align*}
	Then we can write arithmetic-geometric-harmonic mean inequality as $\mathfrak{A}(\mathbf{a})\geq\mathfrak{G}(\mathbf{a})\geq \mathfrak{H}(\mathbf{a})$. This is generalized in the next result.
		\begin{theorem}[Power mean inequality]
			If $r,s$ are real numbers such that $r\leq s$, then for $n$ real numbers $a_{1},\ldots,a_{n}$,
				\begin{align*}
					\min\{\mathbf{a}\}
						& \leq \mathfrak{M}_{r}(\mathbf{a})\leq \mathfrak{M}_{s}(\mathbf{a})\leq\max\{\mathbf{a}\}
				\end{align*}
			Equality occurs if and only if $\mathbf{a}=\mathbf{b}$.
		\end{theorem}
	This inequality can be extended further with the notion of what we call \textit{weighted means}. Let $\omega=(w_{1},\ldots,w_{n})$ be a vector of non-negative real numbers such that $w_{1}+\ldots+w_{n}=1$. Then the weighted arithmetic mean of the real numbers $a_{1},\ldots,a_{n}$ is
		\begin{align*}
			\mathfrak{A}(\mathbf{a},\omega)
				& = w_{1}a_{1}+\ldots+w_{n}a_{n}
		\end{align*}
	In general, the \index{weighted power mean}\textit{weighted power mean of order} $r$ is
		\begin{align*}
			\mathfrak{M}_{r}(\mathbf{a},\omega)
				& = \left(w_{1}a_{1}^{r}+\ldots+w_{n}a_{n}^{r}\right)^{\frac{1}{r}}
		\end{align*}
	If the context is clear on what the weights are, then we may omit the weight from the notation and simply write $\mathfrak{M}_{r}(\mathbf{a})$. Let us call $\omega$ a \textit{weight vector} if $w_{1},\ldots,w_{n}\geq 0$ and $w_{1}+\ldots+w_{n}=1$. We can convert almost any vector of non-negative real numbers into a weight vector. If $\tau=(t_{1},\ldots,t_{n})$ is an arbitrary vector not all elements zero, then
		\begin{align*}
			\omega
				& = \left(\dfrac{t_{1}}{t_{1}+\ldots+t_{n}},\ldots,\dfrac{t_{n}}{t_{1}+\ldots+t_{n}}\right)
		\end{align*}
	is a weight vector. The power mean inequality applies to weighted means as well.
		\begin{theorem}[Weighted Power Mean Inequality]\label{thm:weightedpowermean}
			Let $\mathbf{a}$ and $\omega$ be vectors with $n$ elements. Then for real numbers $r,s$ such that $r\leq s$,
				\begin{align*}
					\min\{\mathbf{a}\} \leq \mathfrak{M}_{r}(\mathbf{a},\omega)
						& \leq \mathfrak{M}_{s}(\mathbf{a},\omega)\leq \max\{\mathbf{a}\}
				\end{align*}
		\end{theorem}
	A special case of this is the \index{weighted arithmetic-geometric mean inequality}\textit{weighted arithmetic-geometric mean inequality}.
		\begin{align}
			w_{1}a_{1}+\ldots+w_{n}a_{n}
				& \geq a_{1}^{w_{1}}\cdots a_{n}^{w_{n}}\label{eqn:weightedamgm}
		\end{align}
\end{document}