\documentclass{subfile}

\begin{document}
	\section{Power Mean Inequality}\label{sec:powermean}
		\begin{definition}[Power mean or Generalized mean]
			Let $a_1,\ldots,a_n$ be positive real numbers. Then the \textit{generalized mean} or \textit{power mean of order} $r$ is defined as
				\begin{align*}
					M_r(a_1,\ldots,a_n)
						& = \left(\dfrac{a_1^r+\ldots+a_n^r}{n}\right)^{\dfrac{1}{r}}
				\end{align*}
			Note that the arithmetic mean of $a_1,\ldots,a_n$ is actually $M_1(a_1,\ldots,a_n)$. Similarly, the harmonic mean is $M_{-1}(a_1,\ldots,a_n)$. Moreover, $M_0(a_1,\ldots,a_n)$ is the geometric mean which we show below.
		\end{definition}
	We may omit the numbers and just call it $M_r$ instead of $M_r(a_1,\ldots,a_n)$ if the context is clear. See the following.
		\begin{align*}
			M_0
				& = \lim\limits_{r\to0}\left(\dfrac{a_1^r+\ldots+a_n^r}{n}\right)^{\dfrac{1}{r}}
		\end{align*}
	
		\begin{theorem}[Power mean inequality]
			If $r,s$ are real numbers such that $r\leq s$, then for $n$ real numbers $a_1,\ldots,a_n$,
				\begin{align*}
					\min(a_1,\ldots,a_n)
						& \leq M_r\leq M_s\leq\max(a_1,\ldots,a_n)
				\end{align*}
		\end{theorem}
\end{document}