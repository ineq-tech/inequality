\documentclass[a4paper, 12pt]{book}

\usepackage{amsmath, amsfonts, amssymb, amsthm}
\usepackage{subfiles, graphicx, enumerate, fancyhdr,titlesec,ifthen,xcolor}
\usepackage{glossaries-extra}
\usepackage[toc,page]{appendix}
\usepackage{setspace}
\usepackage{datetime}
\usepackage{physics}
\usepackage{hyperref}

\usepackage[bibstyle=philosophy-modern,citestyle=philosophy-verbose,backend=biber,autopunct,sortcites,]{biblatex}

\hypersetup{linktocpage,unicode,pdfencoding=auto,}
\DeclareNameAlias{sortname}{given-family}
\interfootnotelinepenalty=10000

\ifxetex
\usepackage[no-math]{fontspec}
\usepackage{unicode-math}
\newcommand{\fakebold}{1}
\newcommand{\scale}{1}
\newcommand{\fakemathbold}{1}
\setmainfont[FakeBold=\fakebold,ItalicFont=ModernMT-ExtendedItalic-New.otf,ItalicFeatures={FakeBold=\fakebold,Scale=\scale},BoldItalicFont=ModernMT-ExtendedItalic-New.otf,SmallCapsFont={OldStandard-Regular.otf},SmallCapsFeatures={Letters=SmallCaps,FakeBold=\fakebold,RawFeature=+smcp,Scale=\scale},BoldFont=OldStandard-Bold.otf,BoldFeatures={FakeBold=0,Scale=\scale,SmallCapsFont=OldStandard-Bold.otf,SmallCapsFeatures={RawFeature=+smcp,Scale=\scale}},BoldItalicFeatures={FakeBold=3},]{ModernMT-Extended-New.otf}
\setsansfont[FakeBold=1,Scale=\scale]{Latin Modern Sans}
\setmonofont[FakeBold=1,Scale=\scale]{Latin Modern Mono}
\setmathfont[FakeBold=\fakemathbold,Scale=\scale]{NewCMMath-Book.otf}
\setmathfont[range={it},FakeBold=\fakemathbold,Scale=\scale]{ModernMT-ExtendedItalic-New.otf}
\setmathfont[range={\partial},FakeBold=\fakebold]{OldStandard-Regular.otf}
\newfontfamily{\bask}{ModernExtT-Theano.otf}
\makeatletter
\RenewDocumentCommand{\sum@}{}{\DOTSB\baskervillesum}
\AtBeginDocument{%
	\RenewDocumentCommand{\sum}{}{\mathop{\sum@}\slimits@}%
}
\NewDocumentCommand{\baskervillesum}{}{%
	\mathchoice
	{\makebaskervillesum{1.8}}% displaystyle
	{\makebaskervillesum{1.2}}% textstyle
	{\makebaskervillesum{1}}% scriptstyle
	{\makebaskervillesum{0.7}}% scriptscriptstyle
}
\NewDocumentCommand{\makebaskervillesum}{m}{%
	\vcenter{\hbox{\scalebox{#1}{\bask Σ}}}%
}
\makeatletter
\RenewDocumentCommand{\prod@}{}{\DOTSB\baskervilleprod}
\AtBeginDocument{%
	\RenewDocumentCommand{\prod}{}{\mathop{\prod@}\slimits@}%
}
\NewDocumentCommand{\baskervilleprod}{}{%
	\mathchoice
	{\makebaskervilleprod{1.5}}% displaystyle
	{\makebaskervilleprod{1.2}}% textstyle
	{\makebaskervilleprod{1}}% scriptstyle
	{\makebaskervilleprod{0.7}}% scriptscriptstyle
}
\NewDocumentCommand{\makebaskervilleprod}{m}{%
	\vcenter{\hbox{\scalebox{#1}{\bask ∏}}}%
}
\newfontfamily\intfont{OldStandard-Italic.otf}
\makeatletter
\NewDocumentCommand{\int@}{}{\DOTSB\baskervilleint}
\AtBeginDocument{%
	\RenewDocumentCommand{\int}{}{\mathop{\int@}\slimits@}%
}
\NewDocumentCommand{\baskervilleint}{}{%
	\mathchoice
	{\makebaskervilleint{1.5}}% displaystyle
	{\makebaskervilleint{1.2}}% textstyle
	{\makebaskervilleint{1}}% scriptstyle
	{\makebaskervilleint{0.7}}% scriptscriptstyle
}
\NewDocumentCommand{\makebaskervilleint}{m}{%
	\vcenter{\hbox{\scalebox{#1}{\intfont ∫}}}%
}
\defaultfontfeatures{Mapping=tex-text,Ligatures=Tex}
\else
%\usepackage{pdfrender, xcolor}
\renewcommand*\rmdefault{cmdh}
\usepackage[T1]{fontenc}
%\pdfrender{StrokeColor=black,TextRenderingMode=2,LineWidth=0.1pt}
%\makeatletter\let\normalrender\PdfRender@NormalColorHook\let\PdfRender@NormalColorHook\@empty\newcommand*{\textnormalrender}[1]{\begingroup\normalrender#1\endgroup}\makeatother
\fi

\renewcommand{\contentsname}{\centering\bfseries\scshape Contents}
\usepackage{tocloft}
\renewcommand{\cftsecfont}{\normalsize\normalfont\scshape}
\renewcommand{\cftchapfont}{\normalsize\normalfont\scshape\bfseries}
\renewcommand{\cftsubsecfont}{\normalsize\normalfont\scshape}
\renewcommand{\cftpartfont}{\normalsize\normalfont\scshape\bfseries}

\usepackage[toc,nopostdot,]{glossaries}
\usepackage{makeidx}

\usepackage{nomencl}
\renewcommand{\nomname}{Notations}
\makenomenclature
\glstoctrue
\makeindex
\makeglossaries

\titleformat{\chapter}[display]{\bfseries\scshape\Huge}{}{1ex}{\centering\titlerule\vspace{1ex}\ifthenelse{\value{chapter}>0}{$\S\S$\thechapter\;}{}}[\vspace{1ex}\titlerule]
\titleformat{\section}[display]{\bfseries\scshape\LARGE}{}{1ex}{\centering\ifthenelse{\value{section}>0}{$\S\S$\thesection\;}{}}[\titlerule]
\titleformat{\subsection}[display]{\bfseries\scshape\Large}{}{1ex}{\centering\ifthenelse{\value{section}>0}{$\S\S\S$\thesection\;}{}}[\titlerule]

\newtheorem{theorem}{\textsc{\textbf{Theorem}}}[chapter]
\theoremstyle{definition}
\newenvironment{definition}[1][]{\par\medskip\noindent \textsc{\textbf{{\ifthenelse{\isempty{#1}}{Definition.}{#1.}}}} \rmfamily}{\medskip}

\newcommand{\lcm}{\mbox{lcm}}
\newcommand{\floor}[1]{\left\lfloor{#1}\right\rfloor}
\newcommand{\ceiling}[1]{\left\lceil{#1}\right\rceil}
\newcommand{\parenthesis}[1]{\left(#1\right)}
\newcommand{\curlybrace}[1]{\left\{#1\right\}}
\newcommand{\squarebracket}[1]{\left[#1\right]}
\newcommand{\rad}{\mbox{rad}}
\newcommand{\bigo}[1]{O\parenthesis{#1}}

\renewcommand*{\chapterautorefname}{Chapter}
\renewcommand*{\sectionautorefname}{Section}
\renewcommand*{\subsectionautorefname}{Subsection}

\renewcommand{\chaptermark}[1]{\markboth{#1}{#1}}
\pagestyle{fancyplain}
\fancyhf{}
\rhead{ \fancyplain{}{\small\textit{{\MyAuthor}}} }
\lhead{ \fancyplain{}{\small$\S$\thechapter. \ \textsc{\leftmark}} }
\rfoot{ \fancyplain{}{\thepage}}

\newtheorem{problem}{\textsc{Problem}}

\theoremstyle{definition}
\newtheorem*{remark}{\textit{Remark}}
\newtheorem*{solution}{\textit{Solution}}

\numberwithin{problem}{chapter}

\usepackage{tikz}
\usetikzlibrary{calc}

\newcommand\HRule{\rule{\textwidth}{1pt}}

\renewcommand{\numberline}[1]{#1~}

\providecommand{\HUGE}{\Huge}
\newlength{\drop}

\DeclareRobustCommand{\cs}[1]{\texttt{\char`\\#1}}
\newlength{\tpheight}\setlength{\tpheight}{0.9\textheight}
\newlength{\txtheight}\setlength{\txtheight}{0.9\tpheight}
\newlength{\tpwidth}\setlength{\tpwidth}{0.9\textwidth}
\newlength{\txtwidth}\setlength{\txtwidth}{0.9\tpwidth}

\renewcommand{\chaptermark}[1]{\markboth{#1}{#1}}
\pagestyle{fancyplain}
\fancyhf{}
\rhead{ \fancyplain{}{\small\textit{{Masum Billal}}} }
\lhead{ \fancyplain{}{\small$\S$\thechapter. \ \textsc{\leftmark}} }
\rfoot{ \fancyplain{}{\thepage}}

\definecolor{platinum}{rgb}{0.9, 0.89, 0.89}
\bibliography{ref.bib}
\subfile{glossary.tex}

\begin{document}
	\frontmatter
	\pagestyle{empty}
	%\pagecolor{platinum}
	\begin{titlepage}
		\drop=0.1\txtheight
		\begin{minipage}[t]{0.05\txtwidth}
			\color{black}
			\rule{6pt}{\txtheight}
		\end{minipage}
		\hspace{0.05\txtwidth}
		\begin{minipage}[t]{2\txtwidth}
			\vspace*{\drop}
			{\Large {\textit{Masum Billal}} %\quad\textit{\&}\quad \textsc{Samin Riasat}
			}
			\\
			\rule{1\txtwidth}{1pt} \par
			\vspace{3\baselineskip}
			{\noindent\bfseries ADVANCES IN OLYMPIAD INEQUALITIES} \par
			\vspace{2\baselineskip}
			{\large\itshape Principles and Techniques for Old and New Problems} \par
			\vspace{6.5\baselineskip}
			{\scshape } \par
			\vspace{0.1\baselineskip}
			{\Large } \par
			\vspace{\baselineskip}
			\rule{\txtwidth}{1pt} \par
			\vspace{\baselineskip}
			{\Large }
		\end{minipage}
		\hfill
	\end{titlepage}

	%\begin{refsection}
		\section*{Preface}

		Inequalities have been extensively studied for at least a couple of centuries. Cauchy was among the first of the mathematicians who had major contributions in this literature. But it was probably not until \textcite{hardy_littlewood_polya_1934} we understood that it could be possible to study inequalities in a more systematic way. Since then a good number of books have discussed different aspects of inequalities for example, \textcite{beckenbach_bellman_1983}.

		\section*{Objectives}
			\begin{itemize}
				\item Dis
			\end{itemize}
		%\printbibliography
	%\end{refsection}
	\newpage
	\tableofcontents
	\mainmatter
	\pagestyle{fancyplain}
	%\begin{refsection}
		\chapter{Classical Inequalities}\label{ch:basics}
		\subfile{intro.tex}
		\subfile{cs.tex}
		\subfile{complex.tex}
		\subfile{powermean.tex}
		\subfile{holdmink.tex}
		\subfile{rearrangement.tex}
		\subfile{convexity.tex}
		%\printbibliography
	%\end{refsection}

	%\begin{refsection}
		\chapter{Traditional Principles}\label{ch:traditional}
		\subfile{engel.tex}
		\subfile{buffalo.tex}
		\subfile{majorization.tex}
		\subfile{bunching.tex}
		\subfile{homonorm.tex}
		\subfile{substitution.tex}
		%\printbibliography
	%\end{refsection}

	%\begin{refsection}
		\chapter{Advanced Techniques \& Inequalities}\label{ch:advanced}
		\subfile{tangent.tex}
		\subfile{sv.tex}
		\subfile{radon.tex}
		\subfile{smoothfudging.tex}
		\subfile{uvw.tex}
		\subfile{mv.tex}
		\subfile{miscellaneous.tex}
		\subfile{dumbassing.tex}
		\subfile{ev.tex}
		\subfile{prob.tex}
		%\printbibliography
	%\end{refsection}

	%\begin{refsection}
		\subfile{exercise.tex}
	%\end{refsection}

	%\begin{refsection}
		\subfile{problems.tex}
		\subfile{abmo.tex}
		\subfile{almo.tex}
		\subfile{amc.tex}
		\subfile{apc.tex}
		\subfile{apmo.tex}
		\subfile{azno.tex}
		\subfile{bkmo.tex}
		\subfile{bmo.tex}
		\subfile{bno.tex}
		\subfile{brno.tex}
		\subfile{buno.tex}
		\subfile{chmo.tex}
		\subfile{imo.tex}
		\subfile{irmo.tex}
		\subfile{mmo.tex}
		\subfile{pmo.tex}
		\subfile{rno.tex}
		\subfile{sgmo.tex}
		\subfile{tno.tex}
		\subfile{usamo.tex}
		%\printbibliography
	%\end{refsection}
	\backmatter
	\printbibliography[heading=bibnumbered]
	\appendix
	\printglossaries
	\printindex
\end{document}