\documentclass[a4paper, 12pt]{book}

\usepackage{amsmath, amsfonts, amssymb, amsthm}
\usepackage{subfiles, pdfrender, enumerate, fancyhdr}
\usepackage{csquotes, xcolor}
%\usepackage{baskervillef}
\usepackage[baskerville]{newtxmath}

%\pdfrender{StrokeColor=black,TextRenderingMode=2,LineWidth=0.4pt}
\usepackage[backend=biber,style=alphabetic,natbib=true,sorting=ynt,]{biblatex}


\usepackage[
	colorlinks,
	citecolor=black,
	filecolor=black,
	linkcolor=black,
	urlcolor=black,
	pdfstartview=,
	pdfencoding=auto,
]{hyperref}

\setlength{\textwidth}{6.5in}
\setlength{\oddsidemargin}{.1in}
\setlength{\evensidemargin}{.1in}
\setlength{\topmargin}{-.1in}
\setlength{\textheight}{8.4in}


\setlength{\textwidth}{6.5in}
\setlength{\oddsidemargin}{.1in}
\setlength{\evensidemargin}{.1in}
\setlength{\topmargin}{-.1in}
\setlength{\textheight}{8.4in}

\DeclareMathOperator{\rad}{rad}
\DeclareMathOperator{\lcm}{lcm}
\DeclareMathOperator{\ord}{ord}

\newenvironment{dedication}
{\clearpage           % we want a new page
	\thispagestyle{empty}% no header and footer
	\vspace*{\stretch{1}}% some space at the top
	%\slshape% the text is in italics
	\raggedleft          % flush to the right margin
}
{\par % end the paragraph
	\vspace{\stretch{3}} % space at bottom is three times that at the top
	\clearpage           % finish off the page
}

\pagestyle{fancyplain}
\fancyhf{}
\setlength{\headheight}{52pt}
\fancyhead[RO]{\textit{\nouppercase{\rightmark}}}
\fancyhead[LO, RE]{Masum Billal, Samin Riasat}
\fancyhead[LE]{\textit{\nouppercase{\leftmark}}}
\fancyfoot[LE, RO]{Page\ \thepage}
\renewcommand{\headrulewidth}{.01pt} % remove lines as well
\renewcommand{\footrulewidth}{.01pt}

\newtheorem{theorem}{Theorem}
\newtheorem{lemma}{Lemma}

\theoremstyle{definition}
\newtheorem{definition}{Definition}
\newtheorem{problem}{Problem}
\newtheorem*{solution}{Solution}

\numberwithin{problem}{chapter}

\renewcommand{\thechapter}{\Roman{chapter}}

\title{\bfseries Techniques in Olympiad Inequalities}
\author{Masum Billal\and Samin Riasat}

\begin{document}
	\frontmatter
	\maketitle
	\section*{Preface}
	\tableofcontents
	\mainmatter
	
	\begin{refsection}
		\subfile{basics.tex}
		\subfile{intro.tex}
		\subfile{cs.tex}
		\subfile{rearrangement.tex}
		\subfile{majorization.tex}
		\subfile{powermean.tex}
		\printbibliography
	\end{refsection}
	
	\begin{refsection}
		\subfile{buffalo.tex}
		\printbibliography
	\end{refsection}

	\begin{refsection}
		\subfile{substitution.tex}
		\printbibliography
	\end{refsection}

	\begin{refsection}
		\subfile{homonorm.tex}
		\printbibliography
	\end{refsection}
	
	\begin{refsection}
		\subfile{engel.tex}
		\printbibliography
	\end{refsection}

	\begin{refsection}
		\subfile{reverse.tex}
		\printbibliography
	\end{refsection}

	\begin{refsection}
		\subfile{sos.tex}
		\printbibliography
	\end{refsection}
	
	\begin{refsection}
		\subfile{uvw.tex}
		\printbibliography
	\end{refsection}

	\begin{refsection}
		\subfile{mv.tex}
		\printbibliography
	\end{refsection}

	\begin{refsection}
		\subfile{sv.tex}
		\printbibliography
	\end{refsection}

	\begin{refsection}
		\subfile{dumbassing.tex}
		\printbibliography
	\end{refsection}

	\begin{refsection}
		\subfile{ev.tex}
		\printbibliography
	\end{refsection}

	\begin{refsection}
		\subfile{problems.tex}
		\printbibliography
	\end{refsection}

	\begin{refsection}
		\subfile{exercises.tex}
		\printbibliography
	\end{refsection}
	\backmatter
\end{document}