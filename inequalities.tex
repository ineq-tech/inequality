\documentclass[a4paper, 12pt, leqno, twoside]{book}

\usepackage{amsmath, amsfonts, amssymb, amsthm}
\usepackage{subfiles, graphicx, enumerate, fancyhdr}
\usepackage{glossaries-extra}
\usepackage[toc]{glossaries}
\usepackage[toc,page]{appendix}
\usepackage{makeidx}
\usepackage{setspace}
\usepackage{datetime}
\usepackage{physics}
\usepackage{hyperref}
\hypersetup{
	pdfstartview=,
	pdfencoding=auto,
}
\usepackage{tikz}
\usetikzlibrary{calc}

\newcommand\HRule{\rule{\textwidth}{1pt}}

\renewcommand{\numberline}[1]{#1~}

\providecommand{\HUGE}{\Huge}
\newlength{\drop}

\DeclareRobustCommand{\cs}[1]{\texttt{\char`\\#1}}
\newlength{\tpheight}\setlength{\tpheight}{0.9\textheight}
\newlength{\txtheight}\setlength{\txtheight}{0.9\tpheight}
\newlength{\tpwidth}\setlength{\tpwidth}{0.9\textwidth}
\newlength{\txtwidth}\setlength{\txtwidth}{0.9\tpwidth}

\ifxetex
%\usepackage{mathspec}
\usepackage[no-math]{fontspec}
\usepackage{unicode-math}
\else
\usepackage[LGR,T2A,LY1]{fontenc}
%\usepackage{OldStandard}
%\usepackage[noamssymbols,baskervillef]{newtxmath}
\usepackage{mlmodern}
%\usepackage[activate={true,nocompatibility},final,tracking=true,kerning=true,spacing=true,]{microtype}
%\usepackage{pdfrender, xcolor}
\fi


\ifpdftex
%\microtypecontext{spacing=nonfrench}
%\pdfrender{StrokeColor=black,TextRenderingMode=2,LineWidth=0.1pt}
%\makeatletter\let\normalrender\PdfRender@NormalColorHook\let\PdfRender@NormalColorHook\@empty\newcommand*{\textnormalrender}[1]{\begingroup\normalrender#1\endgroup}\makeatother
\else
\newfontfamily\headingfont[]{BaskervilleF-BoldItalic.otf}
%\usepackage{titlesec}
%\titleformat{\chapter}[display]{\huge\headingfont}{\chaptertitlename\ \thechapter}{20pt}{\Huge\headingfont}
%\titleformat*{\section}{\Large\headingfont}
\newcommand{\fakebold}{2}
\setmainfont[FakeBold=\fakebold,ItalicFont=ModernMT-ExtendedItalic.otf,BoldItalicFont=OldStandard-BoldItalic.otf,BoldFont=OldStandard-Bold.otf,BoldFeatures={FakeBold=0},BoldItalicFeatures={FakeBold=0}]{OldStandard-Regular.otf}
\setmathfont[FakeBold=\fakebold,]{NewCMMath-Book.otf}
\setmathfont[range=it,FakeBold=\fakebold]{Old Standard Italic}
\setmathfont[range={\symfrak},FakeBold=\fakebold]{Asana Math}
\setmathfont[range={\int},Scale=2,]{Old Standard Italic}
\setmathfont[range={\sum,\prod},Scale=2]{Old Standard}

\usepackage{mathspec}
%\setmathsfont(Digits)[Scale=MatchUppercase,FakeBold=\fakebold]{Old Standard}
\setmathsfont(Latin)[Uppercase=Italic,Lowercase=Italic,FakeBold=\fakebold,Scale=MatchUppercase]{ModernMT-ExtendedItalic.otf}
\defaultfontfeatures{Mapping=tex-text,Ligatures=Tex}
\fi

\newtheorem{theorem}{\textsc{Theorem}}
\newtheorem{lemma}{\textsc{Lemma}}
\newtheorem{definition}{\textsc{Definition}}
\newtheorem{problem}{\textsc{Problem}}

\theoremstyle{definition}
\newtheorem*{remark}{\textit{Remark}}
\newtheorem*{solution}{\textit{Solution}}

\usepackage[style=philosophy-verbose,backend=biber,sortcites=true,classical=true,scauthors=true, scauthorscite=true,scauthorsbib=true,volnumformat=strings,volumeformat=romansc,]{biblatex}

\renewcommand{\theequation}{$\ddagger$\ \thechapter.\arabic{equation}}

\numberwithin{problem}{chapter}

\subfile{glossary.tex}
\bibliography{ref.bib}

\makenoidxglossaries
\makeindex

\newcommand*{\problemautorefname}{\textit{Problem}}

\renewcommand*{\chapterautorefname}{$\S$}
\renewcommand*{\sectionautorefname}{$\S$}
\renewcommand*{\subsectionautorefname}{$\S$}

\pagestyle{fancyplain}
\fancyhf{}
\setlength{\headheight}{52pt}

\fancyhead[LO, RE]{\small\textit{Masum Billal}\textit{\&}%\textit{Samin Riasat}
}
\fancyhead[LE]{\small\textit{\nouppercase{\leftmark}}}
\fancyhead[RO]{\small\textit{\nouppercase{\rightmark}}}
\fancyfoot[LE, RO]{Page\ \thepage}
\renewcommand{\headrulewidth}{.01pt} % remove lines as well
\renewcommand{\footrulewidth}{.01pt}

\begin{document}
	\frontmatter
	\begin{titlepage}
		\drop=0.1\txtheight
		\begin{minipage}[t]{0.05\txtwidth}
			\color{black}
			\rule{6pt}{\txtheight}
		\end{minipage}
		\hspace{0.05\txtwidth}
		\begin{minipage}[t]{2\txtwidth}
			\vspace*{\drop}
			{\Large {\scshape Masum Billal} %\quad\textit{\&}\quad \textsc{Samin Riasat}
			} 
			\\
			\rule{1\txtwidth}{1pt} \par
			\vspace{3\baselineskip}
			{\noindent\bfseries ADVANCES IN OLYMPIAD INEQUALITIES} \par
			\vspace{2\baselineskip}
			{\large\itshape Principles and Techniques for Old and New Problems} \par
			\vspace{6.5\baselineskip}
			{\scshape } \par
			\vspace{0.1\baselineskip}
			{\Large } \par
			\vspace{\baselineskip}
			\rule{\txtwidth}{1pt} \par
			\vspace{\baselineskip}
			{\Large THE PUBLISHER}
		\end{minipage}
		\hfill
	\end{titlepage}

	\begin{refsection}
		\section*{Preface}

		Inequalities have been extensively studied for at least a couple of centuries. Cauchy was among the first of the mathematicians who had major contributions in this literature. But it was probably not until \textcite{hardy_littlewood_polya_1934} we understood that it could be possible to study inequalities in a more systematic way. Since then a good number of books have discussed different aspects of inequalities for example, \textcite{beckenbach_bellman_1983}.

		\section*{Objectives}
			\begin{itemize}
				\item Dis
			\end{itemize}
		\printbibliography
	\end{refsection}
	\tableofcontents
	\mainmatter

	\begin{refsection}
		\chapter{Classical Inequalities}\label{ch:basics}
		\subfile{intro.tex}
		\subfile{cs.tex}
		\subfile{complex.tex}
		\subfile{powermean.tex}
		\subfile{holdmink.tex}
		\subfile{rearrangement.tex}
		\subfile{convexity.tex}
		\printbibliography
	\end{refsection}

	\begin{refsection}
		\chapter{Traditional Principles}\label{ch:traditional}
		\subfile{engel.tex}
		\subfile{buffalo.tex}
		\subfile{majorization.tex}
		\subfile{bunching.tex}
		\subfile{homonorm.tex}
		\subfile{substitution.tex}
		\printbibliography
	\end{refsection}

	\begin{refsection}
		\chapter{Advanced Techniques \& Inequalities}\label{ch:advanced}
		\subfile{tangent.tex}
		\subfile{sv.tex}
		\subfile{radon.tex}
		\subfile{smoothfudging.tex}
		\subfile{uvw.tex}
		\subfile{mv.tex}
		\subfile{miscellaneous.tex}
		\subfile{dumbassing.tex}
		\subfile{ev.tex}
		\subfile{prob.tex}
		\printbibliography
	\end{refsection}

	\begin{refsection}
		\subfile{exercise.tex}
	\end{refsection}

	\begin{refsection}
		\subfile{problems.tex}
		\subfile{abmo.tex}
		\subfile{almo.tex}
		\subfile{amc.tex}
		\subfile{apc.tex}
		\subfile{apmo.tex}
		\subfile{azno.tex}
		\subfile{bkmo.tex}
		\subfile{bmo.tex}
		\subfile{bno.tex}
		\subfile{brno.tex}
		\subfile{buno.tex}
		\subfile{chmo.tex}
		\subfile{imo.tex}
		\subfile{irmo.tex}
		\subfile{mmo.tex}
		\subfile{pmo.tex}
		\subfile{rno.tex}
		\subfile{sgmo.tex}
		\subfile{tno.tex}
		\subfile{usamo.tex}
		\printbibliography
	\end{refsection}

	\backmatter
	\appendix
	\printnoidxglossary[]
	\printindex
\end{document}