\documentclass[a4paper, 12pt, leqno, twoside]{book}

\usepackage{amsmath, amsfonts, amssymb, amsthm}
\usepackage{subfiles, graphicx, enumerate, fancyhdr}
\usepackage{csquotes, xcolor, titlesec}
\usepackage{glossaries-extra}
\usepackage[toc]{glossaries}
\usepackage[toc,page]{appendix}
\usepackage{makeidx}
\usepackage{setspace}
\usepackage{afterpage}
\usepackage{datetime}
\usepackage{physics}

\usepackage[no-math]{fontspec}
%\usepackage{mathspec}

\setmainfont{ModernMTStd-Extended.otf}[
	FakeBold=2,
	SmallCapsFont=MrsEavesSmallCaps_Regular.ttf,
	BoldFont=goodfish-rg.ttf,
	ItalicFont=ModernMT-ExtendedItalic.otf,
	BoldItalicFont=ModernMT-ExtendedItalic.otf,	Ligatures=TeX,]

\setsansfont[%
FakeBold=2,ItalicFont=NewCMSans10-Oblique.otf,BoldFont=NewCMSans10-Bold.otf,BoldItalicFont=NewCMSans10-BoldOblique.otf,
SmallCapsFeatures={Numbers=OldStyle}]{NewCMSans10-Regular.otf}

\setmonofont[%
FakeBold=2,ItalicFont=NewCMMono10-Italic.otf,BoldFont=NewCMMono10-Bold.otf,
BoldItalicFont=NewCMMono10-BoldOblique.otf,SmallCapsFeatures={Numbers=OldStyle}]{NewCMMono10-Regular.otf}

\usepackage{unicode-math}
\setmathfont{NewCMMath-Book.otf}[FakeBold=2]


%\defaultfontfeatures{Mapping=tex-text, Numbers=OldStyle}
%\setmainfont[FakeBold=2,BoldFont=ModernMTStd-Bold.otf,ItalicFont=ModernMT-ExtendedItalic.otf,SmallCapsFont=MrsEavesSmallCaps_Regular.ttf]{ModernMTStd-Extended.otf}
%\setmathrm[FakeBold=2]{baskervillef}
%\setmathsfont(Greek)[Uppercase=Plain,Lowercase=Regular,FakeBold=2]{baskervillef}
%\setmathsfont(Digits)[Numbers=OldStyle,FakeBold=2]{baskervillef} % because      Solomos' "1" wasn't as good
%\setmathsfont(Latin)[Uppercase=Italic,Lowercase=Italic,FakeBold=2]{baskervillef}

%\setallsansfonts[Numbers={OldStyle,Proportional},Scale=MatchLowercase,FakeBold=2]{Old Standard}
%\setallmonofonts[Numbers=OldStyle,Scale=MatchLowercase,FakeBold=2]{Courier New}
%\setmathsfont(Digits,Latin)[Scale=MatchLowercase]{Bembo MT}

\newfontfamily{\bask}{GFS Baskerville}
\let\sum\relax
\DeclareMathOperator*{\sum}{\raisebox{-3.5pt}{\scalebox{2}{\rotatebox{1}{{\bask Σ}}}}}


\usepackage{hyperref}
\usepackage{mystyle}

\newtheorem{theorem}{\textsc{Theorem}}
\newtheorem{lemma}{\textsc{Lemma}}
\newtheorem{definition}{\textsc{Definition}}
\newtheorem{problem}{\textsc{Problem}}

\theoremstyle{definition}
\newtheorem*{remark}{\textit{Remark}}
\newtheorem*{solution}{\textit{Solution}}

\makeatletter
\newcommand{\globalcolor}[1]{%
	\color{#1}\global\let\default@color\current@color
}
\makeatother
\AtBeginDocument{\globalcolor{smokyblack}}

\titleformat{\chapter}{\centering\LARGE\bfseries\color{deepjunglegreen}}{$\S$\LARGE\thechapter.}{0.5em}{}[{\titlerule[02pt]}]
\titleformat{\section}{\centering\Large\bfseries\color{deepjunglegreen}}{$\S$\Large\thesection.}{0.5em}{}[{\titlerule[01pt]}]
\titleformat{\subsection}{\centering\large\bfseries\color{deepjunglegreen}}{$\S$\large\thesubsection.}{0.5em}{}[{\titlerule[0.5pt]}]

\usepackage[backend=biber,style=alphabetic,natbib=true,sorting=nyt,]{biblatex}

\hypersetup{
	colorlinks,
	citecolor=deepjunglegreen,
	filecolor=deepjunglegreen,
	linkcolor=deepjunglegreen,
	urlcolor=deepjunglegreen,
	pdfstartview=,
	pdfencoding=auto,
}

\renewcommand{\theequation}{$\ddagger$\ \thechapter.\arabic{equation}}

\numberwithin{problem}{chapter}

\subfile{glossary.tex}
\bibliography{ref.bib}

\makenoidxglossaries
\makeindex

\newcommand*{\problemautorefname}{\textit{Problem}}

\renewcommand*{\chapterautorefname}{$\S$}
\renewcommand*{\sectionautorefname}{$\S$}
\renewcommand*{\subsectionautorefname}{$\S$}


\pagestyle{fancyplain}
\fancyhf{}
\setlength{\headheight}{52pt}

\fancyhead[LO, RE]{\small\textit{Masum Billal}\textit{\&}\textit{Samin Riasat}}
\fancyhead[LE]{\small\textit{\nouppercase{\leftmark}}}
\fancyhead[RO]{\small\textit{\nouppercase{\rightmark}}}
\fancyfoot[LE, RO]{Page\ \thepage}
\renewcommand{\headrulewidth}{.01pt} % remove lines as well
\renewcommand{\footrulewidth}{.01pt}

\begin{document}
	\frontmatter
	\begin{titlepage}
		\pagecolor{splashedwhite}
		%\afterpage{splashedwhite}
		\drop=0.1\txtheight
		\begin{minipage}[t]{0.05\txtwidth}
			\color{black}
			\rule{6pt}{\txtheight}
		\end{minipage}
		\hspace{0.05\txtwidth}
		\begin{minipage}[t]{2\txtwidth}
			\color{smokyblack}
			\vspace*{\drop}
			{\Large {\scshape Masum Billal} \quad\textit{\&}\quad \textsc{Samin Riasat}} \\
			\rule{1\txtwidth}{1pt} \par
			\vspace{3\baselineskip}
			{\noindent\bfseries ADVANCES IN OLYMPIAD INEQUALITIES} \par
			\vspace{2\baselineskip}
			{\large\itshape Principles and Techniques for Old and New Problems} \par
			\vspace{6.5\baselineskip}
			{\scshape } \par
			\vspace{0.1\baselineskip}
			{\Large } \par
			\vspace{\baselineskip}
			\rule{\txtwidth}{1pt} \par
			\vspace{\baselineskip}
			{\Large THE PUBLISHER}
		\end{minipage}
		\hfill
	\end{titlepage}

	\begin{refsection}
		\section*{Preface}

		Inequalities have been extensively studied for at least a couple of centuries. Cauchy was among the first of the mathematicians who had major contributions in this literature. But it was probably not until \textcite{hardy_littlewood_polya_1934} we understood that it could be possible to study inequalities in a more systematic way. Since then a good number of books have discussed different aspects of inequalities for example, \textcite{beckenbach_bellman_1983}.

		\section*{Objectives}
			\begin{itemize}
				\item Dis
			\end{itemize}
		\printbibliography
	\end{refsection}
	\tableofcontents
	\mainmatter

	\begin{refsection}
		\chapter{Classical Inequalities}\label{ch:basics}
		\subfile{intro.tex}
		\subfile{cs.tex}
		\subfile{complex.tex}
		\subfile{powermean.tex}
		\subfile{holdmink.tex}
		\subfile{rearrangement.tex}
		\subfile{convexity.tex}
		\printbibliography
	\end{refsection}

	\begin{refsection}
		\chapter{Traditional Principles}\label{ch:traditional}
		\subfile{engel.tex}
		\subfile{buffalo.tex}
		\subfile{majorization.tex}
		\subfile{bunching.tex}
		\subfile{homonorm.tex}
		\subfile{substitution.tex}
		\printbibliography
	\end{refsection}

	\begin{refsection}
		\chapter{Advanced Techniques \& Inequalities}\label{ch:advanced}
		\subfile{tangent.tex}
		\subfile{sv.tex}
		\subfile{radon.tex}
		\subfile{smoothfudging.tex}
		\subfile{uvw.tex}
		\subfile{mv.tex}
		\subfile{miscellaneous.tex}
		\subfile{dumbassing.tex}
		\subfile{ev.tex}
		\subfile{prob.tex}
		\printbibliography
	\end{refsection}

	\begin{refsection}
		\subfile{exercise.tex}
	\end{refsection}

	\begin{refsection}
		\subfile{problems.tex}
		\subfile{abmo.tex}
		\subfile{almo.tex}
		\subfile{amc.tex}
		\subfile{apc.tex}
		\subfile{apmo.tex}
		\subfile{azno.tex}
		\subfile{bkmo.tex}
		\subfile{bmo.tex}
		\subfile{bno.tex}
		\subfile{brno.tex}
		\subfile{buno.tex}
		\subfile{chmo.tex}
		\subfile{imo.tex}
		\subfile{irmo.tex}
		\subfile{mmo.tex}
		\subfile{pmo.tex}
		\subfile{rno.tex}
		\subfile{sgmo.tex}
		\subfile{tno.tex}
		\subfile{usamo.tex}
		\printbibliography
	\end{refsection}

	\backmatter
	\appendix
	\printnoidxglossary[]
	\printindex
\end{document}