\documentclass{subfile}

\begin{document}
	\section{Engel Form of Cauchy-Schwarz}\label{sec:engel}
	We mentioned earlier that there is a particular case of Cauchy-Schwarz inequality which is very useful for solving problems. It is known as \index{Engel form of Cauchy-Schwarz}\emph{Engel form of Cauchy-Schwarz}. Some people also call it \index{Titu's lemma}\emph{Titu's lemma} or \emph{T2's lemma}. This name became popular among the USA students who attended the IMO training camp after a lecture given by Titu Andreescu at Math Olympiad Summer Program (MOSP) at Georgetown University in June, $2001$. Even though it is a direct consequence of Cauchy-Schwarz inequality, it can be proven independently as well. Moreover, Cauchy-Schwarz inequality can be proven using this result as well.
		\begin{theorem}[Engel form of Cauchy-Schwarz]\label{thm:engel}
			Let $a_1,\ldots,a_n,b_1,\ldots,b_n$ be real numbers. Then
				\begin{align}
					\dfrac{a_1^2}{x_1}+\ldots+\dfrac{a_n^2}{x_1}
						& \geq\dfrac{(a_1+\ldots+a_n)^2}{x_1+\ldots+x_n}\label{ineq:engel}
				\end{align}
		\end{theorem}
	This inequality has also been called \index{Bergstr\"{o}m's inequality}\emph{Bergstr\"{o}m's inequality}. 
		\begin{proof}[Proof using Cauchy-Schwarz]
			Use Cauchy-Schwarz inequality on
				\begin{align*}
					\frac{a_1}{\sqrt{x_1}},\ldots,\frac{a_1}{\sqrt{x_1}};\sqrt{x_1},\ldots,\sqrt{x_n}
				\end{align*}
			We get
				\begin{align*}
					\left(\dfrac{a_1^2}{\sqrt{x_1}^2}+\ldots+\dfrac{a_n^2}{\sqrt{n}^2}\right)\left(\sqrt{x_1}^2+\ldots+\sqrt{x_n}^2\right)
						& \geq\left(\dfrac{a_1}{\sqrt{x_1}}\sqrt{x_1}+\ldots+\dfrac{a_n}{\sqrt{x_n}}\sqrt{x_n}\right)^2\\
						& = (a_1+\ldots+a_n)^2
				\end{align*}
			This proves the theorem.
		\end{proof}
	
		\begin{proof}[Proof by induction]
			We will use induction to prove \ref{ineq:engel}. The inequality is trivial for $n=1$. For $n=2$,
				\begin{align*}
					\dfrac{a^2}{x}+\dfrac{b^2}{y}
						& \geq\dfrac{(a+b)^2}{x+y}\\
					\iff\dfrac{a^2y+b^2x}{xy}
						& \geq\dfrac{a^2+2ab+b^2}{x+y}\\
					\iff a^2xy+b^2x^2+a^2y^2+b^2xy
						& \geq a^2xy+2abxy+b^2xy\\
					\iff (ay-bx)^2
						& \geq0
				\end{align*}
			This is obviously true. Now, assume that the claim is true for 
		\end{proof}
	We get a very useful result as a corollary of this result.
		\begin{theorem}
			Let $a_1,\ldots,a_n,b_1,\ldots,b_n$ be real numbers. Then
				\begin{align*}
					\dfrac{a_1^2}{b_1^2}+\ldots+\dfrac{a_n^2}{b_n^2}
						& \geq\left(\dfrac{a_1+\ldots+a_n}{b_1+\ldots+b_n}\right)^2
				\end{align*}
		\end{theorem}
	
		\begin{proof}
			By \nameref{thm:engel},
				\begin{align*}
					\dfrac{a_1^2}{b_1^2}+\ldots+\dfrac{a_n^2}{b_n^2}
						& \geq\dfrac{(a_1+\ldots+a_n)^2}{b_1^2+\ldots+b_n^2}\\
						& \geq\dfrac{(a_1+\ldots+a_n)^2}{(b_1+\ldots+b_n)^2}
				\end{align*}
			The last inequality follows from the fact that $b_1^2+\ldots+b_n^2\leq(b_1+\ldots+b_n)^2$.
		\end{proof}
	Another result that is similar to Engel form of Cauchy-Schwarz inequality is the \textit{Beckenbach inequality}.
		\begin{theorem}[Beckenbach inequality]
			Let $x_{1},\ldots,x_{n}$ and $y_{1},\ldots,y_{n}$ be positive real numbers. Then
				\begin{align*}
					\dfrac{\sum_{i=1}^{n}x_{i}^{2}}{\sum_{i=1}^{n}x_{i}}+\dfrac{\sum_{i=1}^{n}y_{i}^{2}}{\sum_{i=1}^{n}y_{i}}
						& \geq \dfrac{\sum_{i=1}^{n}(x_{i}+y_{i})^{2}}{\sum_{i=1}^{n}(x_{i}+y_{i})}
				\end{align*}
		\end{theorem}
	We will start with another proof of Nesbitt's inequality with the Engel form.
		\begin{proof}[Nesbitt inequality using Engel form]
			We will use the fact that $a^2+b^2+c^2\geq ab+bc+ca$.
				\begin{align*}
					S
						& = \dfrac{a}{b+c}+\dfrac{b}{c+a}+\dfrac{c}{a+b}\\
						& = \dfrac{a^2}{ab+ca}+\dfrac{b^2}{bc+ab}+\dfrac{c^2}{ca+bc}\\
						& \geq\dfrac{(a+b+c)^2}{ab+ca+bc+ab+ca+bc}\\
						& = \dfrac{(a+b+c)^2}{2(ab+bc+ca)}\\
						& = \dfrac{a^2+b^2+c^2+2(ab+bc+ca)}{ab+bc+ca}\\
						& = \dfrac{a^2+b^2+c^2}{2(ab+bc+ca)}+1\\
						& \geq\dfrac{1}{2}+1=\dfrac{3}{2}
				\end{align*}
		\end{proof}
	
\end{document}