\documentclass{subfile}

\begin{document}
	\section{Engel Form of Cauchy-Schwarz}\label{sec:engel}
	We mentioned earlier that there is a particular case of Cauchy-Schwarz inequality which is very useful for solving problems. It is known as \index{Engel form of Cauchy-Schwarz}\emph{Engel form of Cauchy-Schwarz}. Some people also call it \index{Titu's lemma}\emph{Titu's lemma} or \emph{T2's lemma}. This name became popular among the USA students who attended the IMO training camp after a lecture given by Titu Andreescu at Math Olympiad Summer Program (MOSP) at Georgetown University in June, $2001$. Even though it is a direct consequence of Cauchy-Schwarz inequality, it can be proven independently as well. Conversely, Cauchy-Schwarz inequality can be proven using this result.
		\begin{theorem}[\itshape Engel form of Cauchy-Schwarz]\label{thm:engel}
			Let $a_1,\ldots,a_n,b_1,\ldots,b_n$ be real numbers. Then
				\begin{align}
					\dfrac{a_1^2}{x_1}+\ldots+\dfrac{a_n^2}{x_1}
						& \geq\dfrac{(a_1+\ldots+a_n)^2}{x_1+\ldots+x_n}\label{ineq:engel}
				\end{align}
		\end{theorem}
	Some writers such as \textcite{mitrinovitch_1959, bellman_1955} also call this \index{Bergstr\"{o}m's inequality}\emph{Bergstr\"{o}m's inequality} due to \textcite{bergstrom_1949}.
		\begin{proof}[Proof using Cauchy-Schwarz]
			Use Cauchy-Schwarz inequality on
				\begin{align*}
					\frac{a_1}{\sqrt{x_1}},\ldots,\frac{a_{n}}{\sqrt{x_{n}}};\sqrt{x_1},\ldots,\sqrt{x_n}
				\end{align*}
			We get
				\begin{align*}
					\left(\dfrac{a_1^2}{\sqrt{x_1}^2}+\ldots+\dfrac{a_n^2}{\sqrt{x_{n}}^2}\right)\left(\sqrt{x_1}^2+\ldots+\sqrt{x_n}^2\right)
						& \geq\left(\dfrac{a_1}{\sqrt{x_1}}\sqrt{x_1}+\ldots+\dfrac{a_n}{\sqrt{x_n}}\sqrt{x_n}\right)^2\\
						& = (a_1+\ldots+a_n)^2
				\end{align*}
			This proves the theorem.
		\end{proof}

		\begin{proof}[Proof by induction]
			We will use induction to prove \ref{ineq:engel}. The inequality is trivial for $n=1$. For $n=2$,
				\begin{align*}
					\dfrac{a^2}{x}+\dfrac{b^2}{y}
						& \geq\dfrac{(a+b)^2}{x+y}\\
					\iff\dfrac{a^2y+b^2x}{xy}
						& \geq\dfrac{a^2+2ab+b^2}{x+y}\\
					\iff a^2xy+b^2x^2+a^2y^2+b^2xy
						& \geq a^2xy+2abxy+b^2xy\\
					\iff (ay-bx)^2
						& \geq0
				\end{align*}
			This is obviously true. Now, assume that the claim is true for
		\end{proof}
	We get a very useful result as a corollary of this result.
		\begin{theorem}
			Let $a_1,\ldots,a_n,b_1,\ldots,b_n$ be real numbers. Then
				\begin{align*}
					\dfrac{a_1^2}{b_1^2}+\ldots+\dfrac{a_n^2}{b_n^2}
						& \geq\left(\dfrac{a_1+\ldots+a_n}{b_1+\ldots+b_n}\right)^2
				\end{align*}
		\end{theorem}

		\begin{proof}
			By \nameref{thm:engel},
				\begin{align*}
					\dfrac{a_1^2}{b_1^2}+\ldots+\dfrac{a_n^2}{b_n^2}
						& \geq\dfrac{(a_1+\ldots+a_n)^2}{b_1^2+\ldots+b_n^2}\\
						& \geq\dfrac{(a_1+\ldots+a_n)^2}{(b_1+\ldots+b_n)^2}
				\end{align*}
			The last inequality follows from the fact that $b_1^2+\ldots+b_n^2\leq(b_1+\ldots+b_n)^2$.
		\end{proof}
	Another result that is similar to \nameref{thm:engel} is a special case of the \textit{Beckenbach inequality}.
		\begin{theorem}
			Let $x_{1},\ldots,x_{n}$ and $y_{1},\ldots,y_{n}$ be positive real numbers. Then
				\begin{align*}
					\dfrac{\sum\limits_{i=1}^{n}x_{i}^{2}}{\sum\limits_{i=1}^{n}x_{i}}+\dfrac{\sum\limits_{i=1}^{n}y_{i}^{2}}{\sum\limits_{i=1}^{n}y_{i}}
						& \geq \dfrac{\sum\limits_{i=1}^{n}(x_{i}+y_{i})^{2}}{\sum\limits_{i=1}^{n}(x_{i}+y_{i})}
				\end{align*}
		\end{theorem}
	We will start with another proof of \nameref{thm:nesbitt} with the \nameref{thm:engel}.
		\begin{proof}[Nesbitt inequality using Engel form]
			We will use the fact that $a^2+b^2+c^2\geq ab+bc+ca$.
				\begin{align*}
					S
						& = \dfrac{a}{b+c}+\dfrac{b}{c+a}+\dfrac{c}{a+b}\\
						& = \dfrac{a^2}{ab+ca}+\dfrac{b^2}{bc+ab}+\dfrac{c^2}{ca+bc}\\
						& \geq\dfrac{(a+b+c)^2}{ab+ca+bc+ab+ca+bc}\\
						& = \dfrac{(a+b+c)^2}{2(ab+bc+ca)}\\
						& = \dfrac{a^2+b^2+c^2+2(ab+bc+ca)}{ab+bc+ca}\\
						& = \dfrac{a^2+b^2+c^2}{2(ab+bc+ca)}+1\\
						& \geq\dfrac{1}{2}+1=\dfrac{3}{2}
				\end{align*}
		\end{proof}
	Let us use this inequality to solve some problems.
		\begin{problem}
			Let $n$ be a positive integer and $a_{1},\ldots,a_{n};b_{1},\ldots,b_{n}$ be positive real numbers. Show that
				\begin{align*}
					(a_{1}b_{1}+\ldots+a_{n}b_{n})\left(\dfrac{a_{1}}{b_{1}}+\ldots+\dfrac{a_{n}}{b_{n}}\right)
						& \geq (a_{1}+\ldots+a_{n})^{2}
				\end{align*}

				\begin{solution}
					By \nameref{thm:engel},
						\begin{align*}
							\dfrac{a_{1}}{b_{1}}+\ldots+\dfrac{a_{n}}{b_{n}}
								& = \dfrac{a_{1}^{2}}{a_{1}b_{1}}+\ldots+\dfrac{a_{n}^{2}}{a_{n}b_{n}}\\
								& \geq \dfrac{(a_{1}+\ldots+a_{n})^{2}}{a_{1}b_{1}+\ldots+a_{n}b_{n}}\\
							\iff (a_{1}b_{1}+\ldots+a_{n}b_{n})\left(\dfrac{a_{1}}{b_{1}}+\ldots+\dfrac{a_{n}}{b_{n}}\right)
								& \geq (a_{1}+\ldots+a_{n})^{2}
						\end{align*}
				\end{solution}
		\end{problem}

		\begin{remark}
			In order to apply the Engel form, we often need transformations like $a/b=a^{2}/b^{2}$. Sometimes we will divide the odd power into an even one and send the remaining one in the denominator, for example,
				\begin{align*}
					\dfrac{a^{2n+1}}{b}
						& = \dfrac{\left(a^{n}\right)^{2}}{\dfrac{b}{a}}
				\end{align*}
			so that we can then apply it. We will demonstrate these ideas in the following problems.
		\end{remark}

		\begin{problem}
			For positive real numbers $a,b,c$, prove that
				\begin{align*}
					\dfrac{a}{b+2c}+\dfrac{b}{c+2a}+\dfrac{c}{a+2b}
						& \geq 1
				\end{align*}

				\begin{solution}
					We again resort to the same trick.
						\begin{align*}
							\dfrac{a}{b+2c}+\dfrac{b}{c+2a}+\dfrac{c}{a+2b}
								& = \dfrac{a^{2}}{ab+2ca}+\dfrac{b^{2}}{bc+2ab}+\dfrac{c^{2}}{ca+2bc}\\
								& \geq \dfrac{(a+b+c)^{2}}{3(ab+bc+ca)}\\
								& = \dfrac{a^{2}+b^{2}+c^{2}+2(ab+bc+ca)}{3(ab+bc+ca)}\\
								& \geq \dfrac{ab+bc+ca+2(ab+bc+ca)}{3(ab+bc+ca)}
						\end{align*}
					In the last line, we used $a^{2}+b^{2}+c^{2}\geq ab+bc+ca$ which is well known by now.
				\end{solution}
		\end{problem}

		\begin{problem}
			Let $x,y,z$ be positive real numbers. Prove that
				\begin{align*}
					\dfrac{2}{x+y}+\dfrac{2}{y+z}+\dfrac{2}{z+x}
						& \geq \dfrac{9}{x+y+z}
				\end{align*}

				\begin{solution}
					This is again a similar problem.
						\begin{align*}
							\dfrac{2}{x+y}+\dfrac{2}{y+z}+\dfrac{2}{z+x}
								& = \dfrac{4}{2(x+y)}+\dfrac{4}{2(y+z)}+\dfrac{4}{2(z+x)}\\
								& \geq \dfrac{(2+2+2)^{2}}{2(x+y+y+z+z+x)}\\
								& = \dfrac{36}{4(x+y+z)}\\
								& = \dfrac{9}{x+y+z}
						\end{align*}
				\end{solution}
		\end{problem}

		\begin{problem}
			Let $x,y,z$ be positive real numbers. Prove that
				\begin{align*}
					\dfrac{x^{2}}{(x+z)(x+y)}+\dfrac{y^{2}}{(y+x)(y+z)}+\dfrac{z^{2}}{(z+y)(z+x)}
						& \geq \dfrac{3}{4}
				\end{align*}

				\begin{solution}
					First, see that $(x+z)(x+y)=x^{2}+xy+zx+yz$. Next, we apply \nameref{thm:engel}.
						\begin{align*}
								& \dfrac{x^{2}}{(x+z)(x+y)}+\dfrac{y^{2}}{(y+x)(y+z)}+\dfrac{z^{2}}{(z+y)(z+x)}\\
								& = \dfrac{x^{2}}{x^{2}+xy+zx+yz}+\dfrac{y^{2}}{y^{2}+yz+xy+zx}+\dfrac{z^{2}}{z^{2}+zx+yz+xy}\\
								& \geq \dfrac{(x+y+z)^{2}}{x^{2}+y^{2}+z^{2}+3(xy+yz+zx)}\\
								& = \dfrac{3}{4}\left(\dfrac{4(x+y+z)^{2}}{3(x^{2}+y^{2}+z^{2}+2(xy+yz+zx)+xy+yz+zx)}\right)\\
								& = \dfrac{3}{4}\left(\dfrac{3(x+y+z)^{2}+(x+y+z)^{2}}{3((x+y+z)^{2}+xy+yz+zx)}\right)\\
								& = \dfrac{3}{4}\left(\dfrac{3(x+y+z)^{2}+x^{2}+y^{2}+z^{2}+2(xy+yz+zx)}{3(x+y+z)^{2}+3(xy+yz+zx)}\right)\\
								& \geq \dfrac{3}{4}\left(\dfrac{3(x+y+z)^{2}+3(xy+yz+zx)}{3(x+y+z)^{2}+3(xy+yz+zx)}\right)\\
								& = \dfrac{3}{4}\cdot1=\dfrac{3}{4}
						\end{align*}
					Thus, the inequality is proved.
				\end{solution}
		\end{problem}

		\begin{problem}[IMO Shortlist $1993$]
			For positive real numbers $a,b,c$ and $d$, prove that
				\begin{align*}
					\dfrac{a}{b+2c+3d}+\dfrac{b}{c+2d+3a}+\dfrac{c}{d+2a+3b}+\dfrac{d}{a+2b+3c}
						& \geq \dfrac{2}{3}
				\end{align*}

				\begin{solution}
					Writing
						\begin{align*}
							\frac{a}{b+2c+3d}
								& =\frac{a^{2}}{ab+2ca+3ad}
						\end{align*}
					and applying \nameref{thm:engel}, we get
						\begin{align*}
							S
								& = \dfrac{a}{b+2c+3d}+\dfrac{b}{c+2d+3a}+\dfrac{c}{d+2a+3b}+\dfrac{d}{a+2b+3c}\\
								& = \dfrac{a^{2}}{ab+2ca+3ad}+\dfrac{b^{2}}{bc+2bd+3ab}+\dfrac{c^{2}}{cd+2ca+3bc}+\dfrac{d^{2}}{ad+2bd+3cd}\\
								& \geq \dfrac{(a+b+c+d)^{2}}{4ab+4bc+4ca+4ad+4bd+4cd}\\
								& = \dfrac{2}{3}\left(\dfrac{3(a+b+c+d)^{2}}{8(ab+bc+ca+ad+bd+cd)}\right)\\
								& = \dfrac{2}{3}\left(\dfrac{3(a+b+c+d)^{2}}{4\left((a+b+c+d)^{2}-(a^{2}+b^{2}+c^{2}+d^{2})\right)}\right)\\
								& = \dfrac{2}{3}\left(\dfrac{4(a+b+c+d)^{2}-(a+b+c+d)^{2}}{4(a+b+c+d)^{2}-4(a^{2}+b^{2}+c^{2})}\right)
						\end{align*}
					By \nameref{thm:cs},
						\begin{align*}
							(a^{2}+b^{2}+c^{2}+d^{2})(1^{2}+1^{2}+1^{2}+1^{2})
								& \geq (a+b+c+d)^{2}\\
							\iff -4(a^{2}+b^{2}+c^{2}+d^{2})
								& \leq -(a+b+c+d)^{2}\\
							\iff 4(a+b+c+d)^{2}-4(a^{2}+b^{2}+c^{2}+d^{2})
								& \leq 4(a+b+c+d)^{2}-(a+b+c+d)^{2}
						\end{align*}
					Thus, we have $S\geq \frac{2}{3}\cdot1=\frac{2}{3}$.
				\end{solution}
		\end{problem}

		\begin{problem}[IMO $1995$, problem $2$]
			Let $a,b,c$ be positive real numbers such that $abc=1$. Prove that
				\begin{align*}
					\dfrac{1}{a^{3}(b+c)}+\dfrac{1}{b^{3}(c+a)}+\dfrac{1}{c^{3}(a+b)}
						& \geq \dfrac{3}{2}
				\end{align*}

				\begin{solution}
					Write
						\begin{align*}
							\frac{1}{a^{3}(b+c)}
								& =\frac{\frac{1}{a^{2}}}{a(b+c)}
						\end{align*}
					and apply \nameref{thm:engel}.
						\begin{align*}
							\dfrac{1}{a^{3}(b+c)}+\dfrac{1}{b^{3}(c+a)}+\dfrac{1}{c^{3}(a+b)}
								& = \dfrac{\dfrac{1}{a^{2}}}{a(b+c)}+\dfrac{\dfrac{1}{b^{2}}}{b(c+a)}+\dfrac{\dfrac{1}{c^{2}}}{c(a+b)}\\
								& \geq \dfrac{\left(\dfrac{1}{a}+\dfrac{1}{b}+\dfrac{1}{c}\right)^{2}}{ab+ca+bc+ab+ca+bc}\\
								& = \dfrac{\left(\dfrac{ab+bc+ca}{abc}\right)^{2}}{2(ab+bc+ca)}\\
								& = \dfrac{ab+bc+ca}{2}\\
								& \geq \dfrac{3\sqrt[3]{ab\cdot bc\cdot ca}}{2}\\
								& = \dfrac{3\sqrt[3]{(abc)^{2}}}{2}\\
								& = \dfrac{3}{2}
						\end{align*}
					Here, we get $ab+bc+ca\geq3\sqrt[3]{ab\cdot bc\cdot ca}$ by \nameref{thm:amgm}.
				\end{solution}
		\end{problem}

		\begin{problem}[Tournament of the Towns, $1998$]
			Let $a,b,c$ be positive real numbers. Prove that
				\begin{align*}
					\dfrac{a^{3}}{a^{2}+ab+b^{2}}+\dfrac{b^{3}}{b^{2}+bc+c^{2}}+\dfrac{c^{3}}{c^{2}+ca+a^{2}}
						& \geq \dfrac{a+b+c}{3}
				\end{align*}

				\begin{solution}
					We again make the numerator a square using
					
						\begin{align*}
							\frac{a^{3}}{a^{2}+ab+b^{2}}
								& =\frac{a^{4}}{a^{3}+a^{2}b+ab^{2}}
						\end{align*}
					Then
						\begin{align*}
							\dfrac{a^{3}}{a^{2}+ab+b^{2}}+\dfrac{b^{3}}{b^{2}+bc+c^{2}}+\dfrac{c^{3}}{c^{2}+ca+a^{2}}
								& = \dfrac{a^{4}}{a^{3}+a^{2}b+ab^{2}}+\dfrac{b^{4}}{b^{3}+b^{2}c+bc^{2}}+\dfrac{c^{4}}{c^{3}+c^{2}a+ca^{2}}\\
								& \geq \dfrac{(a^{2}+b^{2}+c^{2})^{2}}{a^{3}+b^{3}+c^{3}+ab(a+b)+bc(b+c)+ca(c+a)}\\
								& = \dfrac{(a^{2}+b^{2}+c^{2})^{2}}{(a+b+c)(ab+bc+ca)}\\
								& = \dfrac{a^{2}+b^{2}+c^{2}}{a+b+c}
						\end{align*}
					The last line follows since $a^{2}+b^{2}+c^{2}\geq ab+bc+ca$. Then the conclusion follows because by \nameref{thm:cs},
						\begin{align*}
							(a^{2}+b^{2}+c^{2})(1^{2}+1^{2}+1^{2})
								& \geq (a+b+c)^{2}\\
							\iff \dfrac{a^{2}+b^{2}+c^{2}}{a+b+c}
								& \geq \dfrac{a+b+c}{3}
						\end{align*}
				\end{solution}
		\end{problem}
\end{document}