\documentclass[inequalities.tex]{subfile}

\begin{document}
	
	\section{Tangent Line Trick}\label{sec:tangent}
	The \textit{tangent line trick} is a very useful trick which has been around for quite some time. \textcite{li_2006} demonstrates some problems with this trick. Here, we will discuss this in detail and show how it can tackle problems.
	
	Imagine that we are given an inequality in $a_{1},\ldots,a_{n}$. The inequality can be divided into a sum of expressions which is same for $a_{1},\ldots,a_{n}$. That is it can be written as $f(a_{1})+\ldots+f(a_{n})\geq g(a_{1},\ldots,a_{n})$. Now, we typically try to use Jensen's inequality in cases like this. But in many cases,  $f$ is not convex. Even so, we may still be able to prove something like
		\begin{align*}
			f(x)
				& \geq f(\bar{a})+(x-\bar{a})f'(\bar{a})
		\end{align*}
	where $\bar{a}=\frac{a_{1}+\ldots+a_{n}}{n}$. Summing this for $x=a_{1},\ldots,a_{n}$, we have
		\begin{align*}
			f(a_{1})+\ldots+f(a_{n})
				& \geq nf(\bar{a})
		\end{align*}
	The motivation sort of comes from basic calculus. By \textit{wishful thinking}, we may be able to prove that the slope between the points $(x,f(x))$ and $(\bar{a},f(\bar{a}))$ is at least the slope of the tangent line of $f(x)$ at $x=\bar{a}$. This is where the name comes from. This trick is specially useful if you are given the quantity $a_{1}+\ldots+a_{n}$. If the expressions involved are homogeneous, then you do not even need this value. You can use homogeneity to impose conditions such as $a+b+c=1$. For example, see this problem.
		\begin{problem}
			Let $a,b,c$ be positive real numbers. Prove that
				\begin{align*}
					\dfrac{1}{a(b+c)}+\dfrac{1}{b(c+a)}+\dfrac{1}{c(a+b)}
						& \geq \dfrac{27}{2(a+b+c)^{2}}
				\end{align*}
			
				\begin{solution}
					The inequality is homogeneous in $a,b,c$. So without loss of generality, we can normalize the inequality assuming that $a+b+c=1$. Then $0<a,b,c<1$, $\bar{a}=1/3$ and we get the transformation
						\begin{align*}
							\dfrac{1}{a(1-a)}+\dfrac{1}{b(1-b)}+\dfrac{1}{c(1-c)}
								& \geq \dfrac{27}{2}
						\end{align*}
					We define
						\begin{align*}
							f(x)
								& = \dfrac{1}{a(1-a)}
						\end{align*}
					and attempt to prove the inequality that is required by tangent line trick. Since
						\begin{align*}
							f'(x)
								& = \dfrac{2x-1}{x^{2}(1-x)^{2}}
						\end{align*}
					we have $f(1/3)=9/2$ and $f'(1/3)=-27/4$. Thus, if we can prove the inequality
						\begin{align*}
							f(x)
								& \geq \dfrac{9}{2}-\dfrac{27}{4}\left(x-\dfrac{1}{3}\right)
						\end{align*}
					we are done since we will have
						\begin{align*}
							f(a)+f(b)+f(c)
								& \geq 3\cdot f\left(\dfrac{1}{3}\right)\\
								& = \dfrac{27}{2}
						\end{align*}
					So we should check if the tangent line inequality indeed holds.
						\begin{align*}
							\dfrac{1}{x(1-x)}
								& \geq \dfrac{9}{2}-\dfrac{27}{4}\left(x-\dfrac{1}{3}\right)\\
							\iff \dfrac{1}{x(1-x)}
								& \geq \dfrac{54-81x+27}{12}\\
								& \geq \dfrac{27(1-x)}{4}\\
							\iff x(1-x)^{2}
								& \leq \dfrac{4}{27}
						\end{align*}
					We can check if this holds. If $g(x)=x(1-x)^{2}$, then
						\begin{align*}
							g'(x)
								& = -2x(1-x)+(1-x)^{2}\\
								& = (1-x)(1-3x)
						\end{align*}
					We have $g'(x)=0$ if $x\in\{1,1/3\}$ and
						\begin{align*}
							g''(x)
								& = -3(1-x)-(1-3x)\\
								& = 6x-4
						\end{align*}
					Since $0<x<1$ and $g''(1)=2>0$, $g''(1/3)=-2<0$, we have that $g(x)$ is maximum at $x=1/3$ in the interval $(0,1)$. Thus,
						\begin{align*}
							x(1-x)^{2}
								& \leq \dfrac{1}{3}\left(1-\dfrac{1}{3}\right)^{2}\\
								& = \dfrac{27}{4}
						\end{align*}
					We are finally done.
				\end{solution}
			
				\begin{remark}
					Tangent line is a neat trick but it may always not be pretty. But before you go all in, you can easily check if the desired inequality follows from
						\begin{align*}
							f(a_{1})+\ldots+f(a_{n})
								& \geq nf(\bar{a})
						\end{align*}
					or not. If it does, there is a good chance that $f(x)\geq f(\bar{a})+(x-\bar{a})f'(\bar{a})$ holds as well. Another indication that this might work is that equality occurs for $a_{1}=\ldots=a_{n}=\bar{a}$. In order to avoid calculation with fractional values, we could also assume $a_{1}+\ldots+a_{n}=n$ for normalization so $\bar{a}=1$. Also, we may sometimes have to deal with $\leq$ instead of $\geq$. Let us see an example of this type below.
				\end{remark}
		\end{problem}
	
		\begin{problem}
			Given positive real numbers $a,b,c$ such that $a+b+c\geq 3$. Prove that
				\begin{align*}
					\dfrac{1}{a^{2}+b+c}+\dfrac{1}{a+b^{2}+c}+\dfrac{1}{a+b+c^{2}}
						& \leq 1
				\end{align*}
			
				\begin{solution}
					We have equality in the case $a=b=c=1$. So, we may be optimistic that tangent line trick will work here. Using $b+c\geq 3-a$,
						\begin{align*}
							\dfrac{1}{a^{2}+b+c}+\dfrac{1}{a+b^{2}+c}+\dfrac{1}{a+b+c^{2}}
								& \leq \dfrac{1}{a^{2}+3-a}+\dfrac{1}{b^{2}+3-b}+\dfrac{1}{c^{2}+3-c}
						\end{align*}
					So we are done if we can prove that
						\begin{align*}
							\dfrac{1}{a^{2}+3-a}+\dfrac{1}{b^{2}+3-b}+\dfrac{1}{c^{2}+3-c}
								& \leq 1
						\end{align*}
					Let
						\begin{align*}
							f(x)
								& = \dfrac{1}{x^{2}+3-x}
						\end{align*}
					We have that
						\begin{align*}
							f'(x)
								& = \dfrac{1-2x}{(x^{2}+3-x)^{2}}
						\end{align*}
					Since $f(1)=1/3$ and $f'(1)=-1/9$, we need to prove the tangent line inequality
						\begin{align*}
							f(x)
								& \leq \dfrac{1}{3}-\dfrac{1}{9}(x-1)\\
							\iff \dfrac{1}{x^{2}+3-x}
								& \leq \dfrac{4-x}{9}\\
							\iff 5x^{2}+3-7x-x^{3}
								& \geq 0\\
							\iff x^{3}-5x^{2}+7x-3
								& \leq 0
						\end{align*}
					We can see that $(x-3)$ is a factor of this polynomial. So we can factor it easily.
						\begin{align*}
							\iff x^{2}(x-3)-2x(x-3)+x-3
								& \leq 0\\
							\iff (x-3)(x^{2}-2x+1)
								& \leq 0\\
							\iff (3-x)(x-1)^{2}
								& \geq 0
						\end{align*}
					This is obviously true. And summing up the tangent line inequality, we get the conclusion.
				\end{solution}
		\end{problem}
	
		\begin{problem}[USAMO $2003$]\label{prob:usamo2003}
			Let $a,b,c$ be positive real numbers. Prove that
				\begin{align*}
					\dfrac{(2a+b+c)^{2}}{2a^{2}+(b+c)^{2}}+\dfrac{(2b+c+a)^{2}}{2b^{2}+(c+a)^{2}}+\dfrac{(2c+a+b)^{2}}{2c^{2}+(a+b)^{2}}
						& \leq 8
				\end{align*}
			
				\begin{solution}
					Due to homogeneity, we can assume that $0<a,b,c<1$ and $a+b+c=1$ without loss of generality. Then we are required to prove
						\begin{align*}
							\dfrac{(1+a)^{2}}{2a^{2}+(1-a)^{2}}+\dfrac{(1+b)^{2}}{2b^{2}+(1-b)^{2}}+\dfrac{(1+c)^{2}}{2c^{2}+(1-c)^{2}}
								& \leq 8
						\end{align*}
					We can see that equality occurs for $a=b=c=1/3$. Define
						\begin{align*}
							f(x)
								& = \dfrac{(1+x)^{2}}{2x^{2}+(1-x)^{2}}\\
								& = \dfrac{(1+x)^{2}}{3x^{2}-2x+1}
						\end{align*}
					The tangent line inequality in this case is
						\begin{align*}
							f(x)
								& \leq f(\bar{a})+(x-\bar{a})f'(\bar{a})
						\end{align*}
					where
						\begin{align*}
							f'(x)
								& = \dfrac{2(3x^{2}-2x+1)(1+x)-2(1+x)^{2}(3x-1)}{(3x^{2}-2x+1)^{2}}\\
								& = \dfrac{2(1+x)}{3x^{2}-2x+1}-\dfrac{2(1+x)^{2}(3x-1)}{(3x^{2}-2x+1)^{2}}
						\end{align*}
					We have $f(1/3)=8/3$ and $f'(1/3)=4$ so if we can prove that
						\begin{align*}
							f(x)
								& \leq \dfrac{8}{3}+4\left(x-\dfrac{1}{3}\right)\\
								& = \dfrac{4(3x+1)}{3}
						\end{align*}
					for $0<x<1$, then the inequality follows.
						\begin{align*}
							f(x)
								& \leq \dfrac{4(3x+1)}{3}\\
							\iff \dfrac{(1+x)^{2}}{3x^{2}-2x+1}
								& \leq \dfrac{12x+3}{3}\\
							\iff 36x^{3}-15x^{2}-2x+1
									& \geq 0\\
							\iff (3x-1)^{2}(4x+1)
								& \geq 0
						\end{align*}
					This is obviously true. So, we have the inequality.
				\end{solution}
		\end{problem}
\end{document}