\documentclass{subfile}

\begin{document}
	\chapter{Problems}\label{ch:problems}
	Inequalities have been a part of the mathematical contests for a long time. We have borrowed a lot of problems from these prominent contests. Here is a list of the contests (in no particular order):
		\begin{itemize}
			\item \Gls{imo}
			\item \Gls{apmo}
			\item \Gls{brno}
			\item \Gls{buno}
			\item \Gls{chmo}
			\item \Gls{usamo}
			\item \Gls{sgmo}
			\item \Gls{vno}
			\item \Gls{bkmo}
			\item \Gls{amc}
			\item \Gls{apc}
			\item \Gls{bmo}
			\item \Gls{azno}
			\item \Gls{tno}
			\item \Gls{rno}
			\item \Gls{almo}
			\item \Gls{pmo}
			\item \Gls{mmo}
			\item \Gls{abmo}
			\item \Gls{bno}
			\item \Gls{irmo}
		\end{itemize}
	The contests appear in this chapter alphabetically. Problems within a region are ordered based on which year they appeared. In most countries, mathematical Olympiads are followed by team selection tests. However, for better categorization and abbreviation, we have not separated problems based on whether they appeared on a team selection test or national Olympiad. We simply put all the problems under the same region. See \textcite{djukicc_jankovic_matic_2011} for a reference to the IMO problems (at least up to $2009$). See \textcite{dongphd_suugaku_2009} for a reference to the APMO problems, although we should warn the reader that the full book is not written in \LaTeX{} for some reason.
	
	Recall that for a vector $\mathbf{a}=(a_{1},\ldots,a_{n})$, the arithmetic, geometric and harmonic means are $\mathfrak{A}(\mathbf{a}),\mathfrak{G}(\mathbf{a}),\mathfrak{H}(\mathbf{a})$ and $\mathfrak{M}_r(\mathbf{a})$ is the power mean of order $r$. For another vector $\mathbf{b}$, $\langle\mathbf{a},\mathbf{b}\rangle$ is the inner product of $\mathbf{a}$ and $\mathbf{b}$. The $L_{p}$ norm of $\mathbf{a}$ is $\sqrt[p]{a_{1}^{p}+\ldots+a_{n}^{p}}$.
	
\end{document}