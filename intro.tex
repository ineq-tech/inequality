\documentclass{subfile}

\begin{document}
	\section{Introduction}\label{sec:intro}% power mean, cauchy schwarz
	Let us start with the most fundamental inequality.
		\begin{align}
			x^2
				& \geq0\label{eqn:mother}
		\end{align}
	The first author calls it the mother of all inequality. Equality occurs if and only if $x=0$. We can extend it for $n$ variables.
		\begin{align}
			x_1^2+\ldots+x_n^2
				& \geq0\label{eqn:extendedmother}
		\end{align}
	Equality occurs if $x_i=0$ for all $1\leq i\leq n$. We can immediately get some results substituting $x$ with appropriate expressions. Substituting $x$ with $a-b$, we get
		\begin{align*}
			(a-b)^2
				& \geq0\\
			a^2+b^2
				& \geq2ab
		\end{align*}
	This is true for any real numbers $a,b$ and equality occurs if $a=b$. If $a,b$ are positive, then replacing $a$ and $b$ by $\sqrt{a}$ and $\sqrt{b}$ respectively, we get
		\begin{align*}
			a+b
				& \geq2\sqrt{ab}\\
			\iff\sqrt{\dfrac{a}{b}}+\sqrt{\dfrac{b}{a}}
				& \geq2\\
			\iff x+\dfrac{1}{x}
				& \geq2
		\end{align*}
	where $x=\dfrac{a}{b}$.
	
	Note that we can also write the first inequality as
		\begin{align*}
			\dfrac{a+b}{2}
				& \geq\sqrt{ab}
		\end{align*}
	Now, $\frac{a+b}{2}$ is the \textit{arithmetic mean} of $a$ and $b$. On the right side, $\sqrt{ab}$ is the \textit{geometric mean} of $a$ and $b$. So the inequality states that the arithmetic mean of two positive real numbers is greater than or equal to their geometric mean. We can also rewrite it as the following.
		\begin{align*}
			\sqrt{ab}\left(\dfrac{a+b}{2}\right)
				& \geq ab\\
			\sqrt{ab}
				& \geq\dfrac{2ab}{a+b}\\
			\sqrt{ab}
				& \geq\dfrac{2}{\frac{1}{a}+\frac{1}{b}}
		\end{align*}
	$\dfrac{2ab}{a+b}$ is the \textit{harmonic mean} of $a$ and $b$. So, this form of the inequality states that the geometric mean is larger than the harmonic mean. This can be extended for three variables.
		\begin{align*}
			\dfrac{a+b+c}{3}
				& \geq\dfrac{3}{\dfrac{1}{a}+\dfrac{1}{b}+\dfrac{1}{c}}
		\end{align*}
	In fact, this result can be extensively generalized. First, we can consider $n$ variables $a_1,\ldots,a_n$ instead of just $a$ and $b$. Second, we can generalize the fact that $AM\geq GM\geq HM$. We will talk about this generalization in \hyperref[sec:powermean]{$\S\S$\ref*{sec:powermean}}.
	
	Setting $x_1=a-b,x_2=b-c,x_3=c-a$ in inequality \ref{eqn:extendedmother},
		\begin{align*}
			(a-b)^2+(b-c)^2+(c-a)^2
				& \geq0\\
			\iff a^2+b^2+c^2-ab-bc-ca
				& \geq0\\
			\iff a^2+b^2+c^2
				& \geq ab+bc+ca
		\end{align*}
	We could prove this using $a^2+b^2\geq2ab$ repeatedly.
		\begin{align*}
			a^2+b^2
				& \geq2ab\\
			b^2+c^2
				& \geq2bc\\
			c^2+a^2
				& \geq2ca\\
			2(a^2+b^2+c^2)
				& \geq2(ab+bc+ca)\\
			a^2+b^2+c^2-ab-bc-ca
				& \geq0
		\end{align*}
	Multiplying both sides by $(a+b+c)$ and using the fact $a^3+b^3+c^3-3abc=(a+b+c)(a^2+b^2+c^2-ab-bc-ca)$,
		\begin{align*}
			a^3+b^3+c^3-3abc
				& \geq0\\
			a^3+b^3+c^3
				& \geq3abc
		\end{align*}
	Replacing $a^3,b^3,c^3$ by $u,v,w$,
		\begin{align}
			u+v+w
				& \geq3\sqrt[3]{uvw}\label{ineq:amgm3}
		\end{align}
	
		\begin{theorem}[Nesbitt's inequality]\label{thm:nesbitt}
			Let $a,b,c$ be real positive numbers. Then
				\begin{align*}
					\dfrac{a}{b+c}+\dfrac{b}{c+a}+\dfrac{c}{a+b}
						& \geq\dfrac{3}{2}
				\end{align*}
			and equality occurs if and only $a=b=c$.
		\end{theorem}
	
		\begin{proof}[Classical proof]
			First, we will try to \textit{familiarize} the expression on the left side.
				\begin{align}
					S
						& = \dfrac{a}{b+c}+\dfrac{b}{c+a}+\dfrac{c}{a+b}\nonumber\\
						& = \dfrac{a+b+c}{b+c}-1+\dfrac{a+b+c}{c+a}-1+\dfrac{a+b+c}{a+b}-1\nonumber\\
						& = (a+b+c)\left(\dfrac{1}{b+c}+\dfrac{1}{c+a}+\dfrac{1}{a+b}\right)-3\nonumber\\
						& = \dfrac{1}{2}(a+b+b+c+c+a)\left(\dfrac{1}{b+c}+\dfrac{1}{c+a}+\dfrac{1}{a+b}\right)-3\label{ineq:nesbittf}
				\end{align}
			Setting $x=a+b,y=b+c,z=c+a$,
				\begin{align*}
					S
						& = \dfrac{1}{2}(x+y+z)\left(\dfrac{1}{x}+\dfrac{1}{y}+\dfrac{1}{z}\right)-3\\
						& = \dfrac{1}{2}\left(1+\dfrac{x}{y}+\dfrac{x}{z}+\dfrac{y}{x}+1+\dfrac{y}{z}+\dfrac{z}{x}+\dfrac{z}{y}+1\right)-3\\
						& = \dfrac{3}{2}+\dfrac{1}{2}\left(\dfrac{x}{y}+\dfrac{y}{x}+\dfrac{y}{z}+\dfrac{z}{y}+\dfrac{z}{x}+\dfrac{x}{z}\right)-3\\
						& = \dfrac{3}{2}+\dfrac{1}{2}\left(u+\dfrac{1}{u}+v+\dfrac{1}{v}+w+\dfrac{1}{w}\right)-3
				\end{align*}
			where $u=\dfrac{x}{y},v=\dfrac{y}{z},w=\dfrac{z}{x}$. Evidently, $u+\frac{1}{u}\geq2,v+\frac{1}{v}\geq2,w+\frac{1}{w}\geq2$ and we have
				\begin{align*}
					S
						& \geq\dfrac{3}{2}+\dfrac{1}{2}(2+2+2)-3
				\end{align*}
			Thus, $S\geq\frac{3}{2}$. Equality occurs if $u=1,v=1,z=1$ or $x=y=z$ or $a=b=c$.
		\end{proof}
	
		\begin{proof}[Proof by arithmetic-harmonic mean inequality]
			We write the arithmetic-harmonic mean inequality for $u,v,w$ as below.
				\begin{align*}
					\dfrac{u+v+w}{3}
						& \geq\dfrac{3}{\dfrac{1}{u}+\dfrac{1}{v}+\dfrac{1}{v}}\\
					\iff(u+v+w)\left(\dfrac{1}{u}+\dfrac{1}{v}+\dfrac{1}{w}\right)
						& \geq9
				\end{align*}
			Using this on \hyperref[ineq:nesbittf]{inequality $\ddagger$\ref{ineq:nesbittf}},
				\begin{align*}
					S
						& \geq\dfrac{1}{2}\cdot9-2
				\end{align*}
		\end{proof}
	
		\begin{proof}[Another proof]
			Using \hyperref[ineq:amgm3]{inequality $\ddagger$\ref{ineq:amgm3}},
				\begin{align*}
					(a+b)+(b+c)+(c+a)
						& \geq3\sqrt[3]{(a+b)(b+c)(c+a)}\\
					\dfrac{1}{a+b}+\dfrac{1}{b+c}+\dfrac{1}{c+a}
						& \geq3\sqrt[3]{\dfrac{1}{(a+b)(b+c)(c+a)}}\\
					S
						& \geq\dfrac{1}{2}\cdot3\sqrt[3]{(a+b)(b+c)(c+a)}\cdot3\sqrt[3]{\dfrac{1}{(a+b)(b+c)(c+a)}}-2\\
						& \geq\dfrac{9}{2}-2=\dfrac{3}{2}
				\end{align*}
			This again proves the inequality.
		\end{proof}
	We have already showed that $a+b\geq2\sqrt{ab}$ which is a special case of the arithmetic-geometric mean inequality.
		\begin{align*}
			a_1+\ldots+a_n
				& \geq\sqrt{a_1\cdots a_n}
		\end{align*}
		Let us consider the following question. We are given four positive real numbers $a,b,c,d$ such that $S=a+b=c+d$. Which of the products between $ab$ and $cd$ is the smaller one?
		\begin{align*}
			4ab
				& = (a+b)^2-(a-b)^2\\
				& = S^2-(a-b)^2\\
			4cd
				& = (c+d)^2-(c-d)^2\\
				& = S^2-(c-d)^2
		\end{align*}
	As we can see here, the sign in $4ab?4cd$ ($?$ to be replaced by one of $>,<,\geq,\leq$) will be dictated by which of the differences $a-b,c-d$ is smaller. Since $x^2\geq0$, if $(a-b)^2<(c-d)^2$, we have $4ab>4cd$. Now, consider the product $a_1\cdots a_n$. We want to see how the product changes as $a_i$ varies with respect to $\bar{a}=\frac{a_1+\ldots+a_n}{n}$.
	
	If all the $a_i$ are equal to each other, then we have nothing to check. Otherwise, there are at least two positive integers $i$ and $j$ such that $a_i$ and $a_j$ are not equal to $\bar{a}$. Moreover, one of them is greater than $\bar{a}$ and the other is smaller than $\bar{a}$ because all of them cannot be greater (or smaller) than $\bar{a}$. Without loss of generality, assume that $a_1,a_2$ are those two numbers and $a_1=\bar{a}-h,a_2=\bar{a}+k$. Now, consider two other positive numbers $c$ and $d$ which keeps the sum fixed, for example $c=\bar{a},d=\bar{a}+k-h$. We have
		\begin{align*}
			cd
				& = \bar{a}(\bar{a}+k-h)\\
				& = \bar{a}^2+\bar{a}k-\bar{a}h\\
			a_1a_2
				& = (\bar{a}-h)(\bar{a}+k)\\
				& = \bar{a}^2+\bar{a}k-\bar{a}h-hk\\
				& = cd-hk < cd\\
			a_1a_2\cdots a_n
				& < cd\cdots a_n
		\end{align*}
	This basically tells us that we can increase the product further if there are other $a_i$ which are not equal to $\bar{a}$ and the product is maximum when all $a_i$ is equal to $\bar{a}$.
\end{document}