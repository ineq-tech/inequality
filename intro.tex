\documentclass{subfile}

\begin{document}
	\section{Introduction}\label{sec:intro}% power mean, cauchy schwarz
	Let us start with the most fundamental inequality.
		\begin{align}
			x^2
				& \geq0\label{ineq:mother}
		\end{align}
	The first author calls it the mother of all inequality. Equality occurs if and only if $x=0$. We can extend it for $n$ variables.
		\begin{align}
			x_1^2+\ldots+x_n^2
				& \geq0\label{ineq:extendedmother}
		\end{align}
	Equality occurs if $x_i=0$ for all $1\leq i\leq n$. We immediately get some useful results substituting $x$ with appropriate expressions. Substituting $x$ with $a-b$, we get
		\begin{align*}
			(a-b)^2
				& \geq0\\
			a^2+b^2
				& \geq2ab
		\end{align*}
	This is true for any real numbers $a,b$ and equality occurs if $a=b$. If $a,b$ are positive, then replacing $a$ and $b$ by $\sqrt{a}$ and $\sqrt{b}$ respectively, we get
		\begin{align*}
			a+b
				& \geq2\sqrt{ab}\\
			\iff\sqrt{\dfrac{a}{b}}+\sqrt{\dfrac{b}{a}}
				& \geq2\\
			\iff x+\dfrac{1}{x}
				& \geq2
		\end{align*}
	where $x=\dfrac{a}{b}$. This can be generalized to the following result.
		\begin{theorem}[Arithmetic-Geometric Inequality]\label{thm:amgm}
			Let $a_1,\ldots,a_n$ be positive real numbers. Then
				\begin{align*}
					\dfrac{a_1+\ldots+a_n}{n}
						& \geq\sqrt[n]{a_1\cdots a_n}\\
					\iff a_1+\ldots+a_n
						& \geq n\sqrt[n]{a_1\cdots a_n}\\
					\iff \left(\dfrac{a_1+\ldots+a_n}{n}\right)^n
						& \geq a_1\cdots a_n
				\end{align*}
		\end{theorem}
	We will show a classical proof of this result here. The proof is due to \textcite{cauchy_1821}. Later in \hyperref[sec:powermean]{$\S\S$\ref{sec:powermean}}, \hyperref[sec:mazorization]{$\S\S$\ref{sec:mazorization}} we will show more proofs.
		\begin{proof}
			
		\end{proof}
	Note the following.
		\begin{align*}
			\dfrac{a+b}{2}
				& \geq\sqrt{ab}
		\end{align*}
	Now, $\frac{a+b}{2}$ is the \textit{arithmetic mean} of $a$ and $b$. On the right side, $\sqrt{ab}$ is the \textit{geometric mean} of $a$ and $b$. So the inequality states that the arithmetic mean of two positive real numbers is greater than or equal to their geometric mean. We can also rewrite it as the following.
		\begin{align*}
			\sqrt{ab}\left(\dfrac{a+b}{2}\right)
				& \geq ab\\
			\sqrt{ab}
				& \geq\dfrac{2ab}{a+b}\\
			\sqrt{ab}
				& \geq\dfrac{2}{\frac{1}{a}+\frac{1}{b}}
		\end{align*}
	$\dfrac{2ab}{a+b}$ is the \textit{harmonic mean} of $a$ and $b$. So, this form of the inequality states that the geometric mean is larger than the harmonic mean. This can be extended for three variables.
		\begin{align*}
			\dfrac{a+b+c}{3}
				& \geq\dfrac{3}{\dfrac{1}{a}+\dfrac{1}{b}+\dfrac{1}{c}}
		\end{align*}
	In fact, this result can be extensively generalized. First, we can consider $n$ variables $a_1,\ldots,a_n$ instead of just $a$ and $b$. Second, we can generalize the fact that $AM\geq GM\geq HM$. We will talk about this generalization in \hyperref[sec:powermean]{$\S\S$\ref*{sec:powermean}}.
		\begin{problem}
			If $a$ is a real number greater than one, prove that $\log{a}+\log_{a}{e}\geq2$.
		\end{problem}
	Throughout the book, if the base of logarithm is unspecified, then $\log{a}$ shall mean $\log_{e}{a}$. Also, try the next problem in a similar manner.
		\begin{problem}
			Prove the inequality
				\begin{align*}
					\dfrac{x^2}{1+x^4}
						& \leq\dfrac{1}{2}
				\end{align*}
		\end{problem}
	Recall from the definition of $e$ that
		\begin{align*}
			e^x
				& = 1+x+\dfrac{x^2}{2!}+\dfrac{x^3}{3!}+\ldots\\
				& = \sum_{i\geq0}\dfrac{x^i}{i!}
		\end{align*}
	You may also know from elementary differentiation that
		\begin{align*}
			\lim\limits_{n\to\infty}\left(1+\dfrac{1}{n}\right)^n
				& = e
		\end{align*}
	In other words, if $x_n=\left(1+\frac{1}{n}\right)^n$, then $x_n$ converges to $e$. However, we also have the relation that $x_n\leq x_{n+1}$. While we can prove this using induction, we will show a better proof with arithmetic-geometric mean inequality here.
		\begin{problem}
			Let $x_n=\left(1+\frac{1}{n}\right)^n$. Prove that $x_n\leq x_{n+1}$.
		\end{problem}
	
		\begin{solution}
			Apply the arithmetic-geometric mean inequality for $a_1=1,a_2=1+\frac{1}{n},\ldots,a_{n+1}=1+\frac{1}{n}$,
				\begin{align*}
					1+\left(1+\dfrac{1}{n}\right)+\ldots+\left(1+\dfrac{1}{n}\right)
						& \geq(n+1)\sqrt[n+1]{1\cdot\left(1+\dfrac{1}{n}\right)\cdots\left(1+\dfrac{1}{n}\right)}\\
					n+2
						& \geq(n+1)\sqrt[n+1]{\left(1+\dfrac{1}{n}\right)^n}\\
					1+\dfrac{1}{n+1}
						& \geq\sqrt[n+1]{\left(1+\dfrac{1}{n}\right)^n}\\
					\left(1+\dfrac{1}{n+1}\right)^{n+1}
						& \geq\left(1+\dfrac{1}{n}\right)^n
				\end{align*}
		\end{solution}
	Note that we can generalize the idea used in this problem.
		\begin{problem}
			For positive real numbers $x,y$ show that
				\begin{align*}
					\left(\dfrac{x+ny}{n+1}\right)^{n+1}
						& \geq xy^n
				\end{align*}
			Equality occurs only for $x=y$.
		\end{problem}
	Similarly, we can show the following.
		\begin{problem}
			Let $y_n=\left(1+\frac{1}{n}\right)^{n+1}$. Show that $y_n\geq y_{n+1}$.
		\end{problem}
	So $y_n$ is decreasing. Using the two problems above, show that $2<e<4$. In fact, we can prove the following using elementary means
		\begin{align*}
			\lim\limits_{n\to\infty}y_n
				& = \lim\limits_{n\to\infty}x_n=e
		\end{align*}
	
		\begin{problem}
			For a positive integer $n$, show that
				\begin{align*}
					\left(\dfrac{n+1}{2}\right)^n
						& \geq n!
				\end{align*}
			Equality occurs only for $n=1$.
		\end{problem}
	
		\begin{solution}
			We use arithmetic-geometric inequality for $1,2,\ldots,n$.
				\begin{align*}
					1+\ldots+n
						& \geq n\sqrt[n]{1\cdots n}\\
					\dfrac{n(n+1)}{2}\\
						& \geq n\sqrt[n]{n!}\\
					\dfrac{n+1}{2}
						& \geq\sqrt[n]{n!}
				\end{align*}
			Equality occurs if and only if $1=\ldots=n$ which is possible only when $n=1$.
		\end{solution}
	Setting $x_1=a-b,x_2=b-c,x_3=c-a$ in \hyperref[ineq:extendedmother]{\ref{ineq:extendedmother}},
		\begin{align*}
			(a-b)^2+(b-c)^2+(c-a)^2
				& \geq0\\
			\iff a^2+b^2+c^2-ab-bc-ca
				& \geq0\\
			\iff a^2+b^2+c^2
				& \geq ab+bc+ca
		\end{align*}
	We could prove this using $a^2+b^2\geq2ab$ repeatedly.
		\begin{align*}
			a^2+b^2
				& \geq2ab\\
			b^2+c^2
				& \geq2bc\\
			c^2+a^2
				& \geq2ca\\
			2(a^2+b^2+c^2)
				& \geq2(ab+bc+ca)\\
			a^2+b^2+c^2-ab-bc-ca
				& \geq0
		\end{align*}
	Multiplying both sides by $(a+b+c)$ and using the fact $a^3+b^3+c^3-3abc=(a+b+c)(a^2+b^2+c^2-ab-bc-ca)$,
		\begin{align*}
			a^3+b^3+c^3-3abc
				& \geq0\\
			a^3+b^3+c^3
				& \geq3abc
		\end{align*}
	Replacing $a^3,b^3,c^3$ by $u,v,w$,
		\begin{align}
			u+v+w
				& \geq3\sqrt[3]{uvw}\label{ineq:amgm3}
		\end{align}
	Before we go into any theory or method of solving problems, here are some basic tactics which are used very often for solving problems. Let $x,y,z>0$ be real numbers.
		\begin{enumerate}[(i)]
			\item If $\dfrac{1}{x}\geq\dfrac{1}{y}$, then $\dfrac{1}{x-a}\geq\dfrac{1}{y}$ for $a\geq0$. Similarly, $\dfrac{1}{x}\geq\dfrac{1}{y-b}$ for $b\geq0$. They are true because $x-a\leq x$ implies $\dfrac{1}{x-a}\geq\dfrac{1}{x}$ and $y+b\geq y$ implies $\dfrac{1}{y+b}\leq\dfrac{1}{y}$.
			\item Check if you can assume an ordering on the variables. For example, if the inequality is symmetric or cyclic on $x,y,z$, you may possibly assume without loss of generality that $x\geq y\geq z$ or $x\leq y\leq z$. We will discuss more on this in \hyperref[ch:buffalo]{$\S$\ref{ch:buffalo}}.
			\item If you cannot assume an ordering on the variables e.g. $a\geq b\geq c$, can you assume that it has a \textit{maximal element} e.g. $a=\max(a,b,c)$? Sometimes this helps in unexpected ways. See the example below.
			\item See if you can get an \textit{if and only if} way of proving an inequality. We will do this very often. If we can use if and only if (or \textit{iff}), then we are free to prove either the if part or the only if part.
			\item Check if you can get some familiar expressions with some basic manipulations such as making the numerator or denominator equal. For example, see the transformation in \hyperref[eqn:nesbittf]{\ref{eqn:nesbittf}}. This often helps us get a clue on what to do with the inequality.
			\item Does some substitutions such as $x=a+b,y=b+c,z=c+a$ or $x=a-b,y=b-c,z=c-a$ help? One could say we used substitution in our first proof of Nesbitt's inequality. Also, see the transformation $a-c=a-b+b-c$ used in the example below. We will check more on substitutions in \hyperref[ch:subs]{$\S$\ref{ch:subs}}.
			\item Check if the inequality holds even if you put some restrictions on it. For example, you may be allowed to assume that one of the variables is $1$ due to some scaling. We will talk about this in \hyperref[ch:homonorm]{$\S$\ref{ch:homonorm}}.
			\item Can you reduce the number of variables without any assumption? See that we have already used it to prove $a^2+b^2+c^2\geq ab+bc+ca$ in a proof above. Also, see a proof of Nesbitt's inequality below where it is enough to prove the inequality for reduced number of variables.
		\end{enumerate}
	For demonstration purposes, let us prove $a^2+b^2+c^2\geq ab+bc+ca$ again exploiting symmetry. Note that if we let
		\begin{align*}
			f(a,b,c)
				& = a^2+b^2+c^2-ab-bc-ca
		\end{align*}
	then $f(a,b,c)$ is \textit{symmetric} on $a,b,c$. We can verify this by the fact that $f(a,b,c)=f(b,a,c)=f(c,a,b)$ and so on. So, without loss of generality, we assume that $a\geq b\geq c$. Then see the following.
		\begin{align*}
			a^2+b^2+c^2
				& \geq ab+bc+ca\\
			a(a-b)+b(b-c)-c(a-c)
				& \geq0\\
			a(a-b)+b(b-c)-c(a-b+b-c)
				& \geq0\\
			a(a-b)+b(b-c)-c(a-b)-c(b-c)
				& \geq0\\
			(a-c)(a-b)+(b-c)^2
				& \geq0
		\end{align*}
	The last inequality immediately follows from the assumption that $a-c\geq0,a-b\geq0,(b-c)^2\geq0$. Also, note that we did not actually require the condition $a\geq b\geq c$. Just assuming $a=\max(a,b,c)$ was enough in this case to claim that the inequality holds. 
		\begin{theorem}[Nesbitt's inequality]\label{thm:nesbitt}
			Let $a,b,c$ be real positive numbers. Then
				\begin{align*}
					\dfrac{a}{b+c}+\dfrac{b}{c+a}+\dfrac{c}{a+b}
						& \geq\dfrac{3}{2}
				\end{align*}
			and equality occurs if and only $a=b=c$.
		\end{theorem}
	The thing about most inequalities is that they can be solved in more than one ways most of the time. We will  prove this inequality along with some other classical results such as \textit{Cauchy-Schwarz inequality} in more than one ways. Some proofs will be discussed later when we develop some certain techniques. For now, we present some proofs using what we have already developed.
	First, we will try to \textit{familiarize} the expression on the left side.
		\begin{align}
			S
				& = \dfrac{a}{b+c}+\dfrac{b}{c+a}+\dfrac{c}{a+b}\nonumber\\
				& = \dfrac{a+b+c}{b+c}-1+\dfrac{a+b+c}{c+a}-1+\dfrac{a+b+c}{a+b}-1\nonumber\\
				& = (a+b+c)\left(\dfrac{1}{b+c}+\dfrac{1}{c+a}+\dfrac{1}{a+b}\right)-3\nonumber\\
				& = \dfrac{1}{2}(a+b+b+c+c+a)\left(\dfrac{1}{b+c}+\dfrac{1}{c+a}+\dfrac{1}{a+b}\right)-3\label{eqn:nesbittf}
		\end{align}
	
		\begin{proof}[Classical proof]
			Setting $x=a+b,y=b+c,z=c+a$,
				\begin{align*}
					S
						& = \dfrac{1}{2}(x+y+z)\left(\dfrac{1}{x}+\dfrac{1}{y}+\dfrac{1}{z}\right)-3\\
						& = \dfrac{1}{2}\left(1+\dfrac{x}{y}+\dfrac{x}{z}+\dfrac{y}{x}+1+\dfrac{y}{z}+\dfrac{z}{x}+\dfrac{z}{y}+1\right)-3\\
						& = \dfrac{3}{2}+\dfrac{1}{2}\left(\dfrac{x}{y}+\dfrac{y}{x}+\dfrac{y}{z}+\dfrac{z}{y}+\dfrac{z}{x}+\dfrac{x}{z}\right)-3\\
						& = \dfrac{3}{2}+\dfrac{1}{2}\left(u+\dfrac{1}{u}+v+\dfrac{1}{v}+w+\dfrac{1}{w}\right)-3
				\end{align*}
			where $u=\dfrac{x}{y},v=\dfrac{y}{z},w=\dfrac{z}{x}$. Evidently, $u+\frac{1}{u}\geq2,v+\frac{1}{v}\geq2,w+\frac{1}{w}\geq2$ and we have
				\begin{align*}
					S
						& \geq\dfrac{3}{2}+\dfrac{1}{2}(2+2+2)-3
				\end{align*}
			Thus, $S\geq\frac{3}{2}$. Equality occurs if $u=1,v=1,z=1$ or $x=y=z$ or $a=b=c$.
		\end{proof}
	
		\begin{proof}[Proof by arithmetic-harmonic mean inequality]
			We write the arithmetic-harmonic mean inequality for $u,v,w$ as below.
				\begin{align*}
					\dfrac{u+v+w}{3}
						& \geq\dfrac{3}{\dfrac{1}{u}+\dfrac{1}{v}+\dfrac{1}{v}}\\
					\iff(u+v+w)\left(\dfrac{1}{u}+\dfrac{1}{v}+\dfrac{1}{w}\right)
						& \geq9
				\end{align*}
			Using this on \ref{eqn:nesbittf},
				\begin{align*}
					S
						& \geq\dfrac{1}{2}\cdot9-2
				\end{align*}
		\end{proof}
	
		\begin{proof}[Variable reduction proof]
			Let us clear the denominators in the original inequality.
				\begin{align*}
					\dfrac{a}{b+c}+\dfrac{b}{c+a}+\dfrac{c}{a+b}
						& \geq\dfrac{3}{2}\\
					\iff2(a^3+b^3+c^3)
						& \geq a^2b+ab^2+b^2c+bc^2+c^2a+ca^2
				\end{align*}
			Notice the cyclic nature in the expression on the right side. There are $6$ terms on the right side and if we count each of $a^3,b^3,c^3$ twice, there are $6$ terms on the left side as well. So rearranging the inequality above as below
				\begin{align*}
					(a^3+b^3)+(b^3+c^3)+(c^3+a^3)
						& \geq (a^2b+ab^2)+(b^2c+bc^2)+(c^2a+ca^2)
				\end{align*}
			tells us that if we can prove $x^3+y^3\geq x^2y+xy^2$, we will be done if we simply sum them up for $(x,y)=(a,b),(b,c),(c,a)$. Here, we reduced the inequality from $3$ variables to $2$. And fortunately, this inequality is a lot easier to prove.
				\begin{align*}
					x^3+y^3
						& \geq x^2y+xy^2\\
					\iff x(x^2-y^2)-y(x^2-y^2)
						& \geq0\\
					\iff (x-y)(x^2-y^2)
						& \geq0\\
					\iff(x-y)^2(x+y)
						& \geq0
				\end{align*}
			The last inequality is evidently true.
		\end{proof}
	
		\begin{proof}[Another proof]
			Using \hyperref[ineq:amgm3]{inequality \ref{ineq:amgm3}},
				\begin{align*}
					(a+b)+(b+c)+(c+a)
						& \geq3\sqrt[3]{(a+b)(b+c)(c+a)}\\
					\dfrac{1}{a+b}+\dfrac{1}{b+c}+\dfrac{1}{c+a}
						& \geq3\sqrt[3]{\dfrac{1}{(a+b)(b+c)(c+a)}}
				\end{align*}
			We can use this observation on \hyperref[eqn:nesbittf]{\ref{eqn:nesbittf}} and get the following.
				\begin{align*}
					S
						& \geq\dfrac{1}{2}\cdot3\sqrt[3]{(a+b)(b+c)(c+a)}\cdot3\sqrt[3]{\dfrac{1}{(a+b)(b+c)(c+a)}}-2\\
						& \geq\dfrac{9}{2}-2=\dfrac{3}{2}
				\end{align*}
			This again proves the inequality.
		\end{proof}
	We have already showed that $a+b\geq2\sqrt{ab}$ which is a special case of the arithmetic-geometric mean inequality. Let us consider the following question. We are given four positive real numbers $a,b,c,d$ such that $S=a+b=c+d$. Which of the products between $ab$ and $cd$ is the smaller one?
		\begin{align*}
			4ab
				& = (a+b)^2-(a-b)^2\\
				& = S^2-(a-b)^2\\
			4cd
				& = (c+d)^2-(c-d)^2\\
				& = S^2-(c-d)^2
		\end{align*}
	As we can see here, the sign in $4ab?4cd$ ($?$ to be replaced by one of $>,<,\geq,\leq$) will be dictated by which of the differences $a-b,c-d$ is smaller. Since $x^2\geq0$, if $(a-b)^2<(c-d)^2$, we have $4ab>4cd$. Now, consider the product $a_1\cdots a_n$. We want to see how the product changes as $a_i$ varies with respect to $\bar{a}=\frac{a_1+\ldots+a_n}{n}$.
	
	If all the $a_i$ are equal to each other, then we have nothing to check. Otherwise, there are at least two positive integers $i$ and $j$ such that $a_i$ and $a_j$ are not equal to $\bar{a}$. Moreover, one of them is greater than $\bar{a}$ and the other is smaller than $\bar{a}$ because all of them cannot be greater (or smaller) than $\bar{a}$. Without loss of generality, assume that $a_1,a_2$ are those two numbers and $a_1=\bar{a}-h,a_2=\bar{a}+k$. Now, consider two other positive numbers $c$ and $d$ which keeps the sum fixed, for example $c=\bar{a},d=\bar{a}+k-h$. We have
		\begin{align*}
			cd
				& = \bar{a}(\bar{a}+k-h)\\
				& = \bar{a}^2+\bar{a}k-\bar{a}h\\
			a_1a_2
				& = (\bar{a}-h)(\bar{a}+k)\\
				& = \bar{a}^2+\bar{a}k-\bar{a}h-hk\\
				& = cd-hk < cd\\
			a_1a_2\cdots a_n
				& < cd\cdots a_n
		\end{align*}
	This basically tells us that we can increase the product further if there are other $a_i$ which are not equal to $\bar{a}$ and the product is maximum when all $a_i$ is equal to $\bar{a}$.
		\begin{problem}
			Show that for a positive integer $n$,
				\begin{align*}
					n!
						& > \left(\dfrac{n}{4}\right)^n
				\end{align*}
		\end{problem}
	We can easily prove this with induction. This can be improved to $n!>\left(\dfrac{n}{3}\right)^n$. In fact, we can prove the following.
		\begin{align*}
			n!
				& > \left(\dfrac{n}{e}\right)^n
		\end{align*}
	This is another nice result. For example, setting $n=2019$,
		\begin{align*}
			2019!
				& > 673^{2019}
		\end{align*}
	We can even bound $n!$ from both sides with the next result.
		\begin{problem}
			For a positive integer $n$, prove the inequality
				\begin{align*}
					e\left(\dfrac{n+1}{e}\right)^{n+1}
						& > n!>\left(\dfrac{n}{e}\right)^n
				\end{align*}
		\end{problem}
	Equality is not possible because on both sides we have non-integers.
		\begin{definition}
			A function $f$ is \textit{non-decreasing} (\textit{increasing}) on the interval $I$ if for any $a,b\in I$, $(a-b)(f(a)-f(b))\geq0$ ($(a-b)(f(a)-f(b))>0$). Similarly, $f$ is \textit{non-increasing} (\textit{decreasing}) on the interval $I$ if for any $a,b\in I$, $(a-b)(f(a)-f(b))\leq0$ ($(a-b)(f(a)-f(b))<0$). If either of these two conditions apply for $f$, then $f$ is a \textit{monotone function}. We say that $f$ is \textit{monotonic}.
		\end{definition}
	Note that the slope between any two points $(a,f(a))$ and $(b,f(b))$ is $m=\frac{f(a)-f(b)}{a-b}$ and the quantity we have used for the definition is
		\begin{align*}
			(a-b)(f(a)-f(b))
				& = (a-b)^2\dfrac{f(a)-f(b)}{f(a)-f(b)}\\
				& = (a-b)^2m
		\end{align*}
	So, the sign of this quantity is the same as the sign of the slope $m$.
		\begin{definition}
			A sequence $(a_n)$ is \textit{non-decreasing} if $a_i\leq a_{i+1}$ for all $i\in\mathbf{N}$. $(a_n)$ is \textit{strictly increasing} if $a_i<a_{i+1}$. Similarly, $(a_n)$ is \textit{non-increasing} if $a_i\geq a_{i+1}$ for all $i$. $(a_n)$ is \textit{strictly decreasing} if $a_i>a_{i+1}$. If $(a_n)$ is either increasing or decreasing, then $(a_n)$ is a \textit{monotone sequence}. We say that $(a_n)$ is \textit{monotonic}.
		\end{definition}
	
		\begin{theorem}[Abel's Inequality]
			Let $(a_n)$ and $(b_n)$ be two sequences of real numbers such that $b_1\geq \ldots\geq b_n\geq0$. For $1\leq k\leq n$, define
				\begin{align*}
					s_k
						& = a_1+\ldots+a_k\\
					m
						& = \min_{1\leq i\leq n}(s_i)\\
					M
						& = \max_{1\leq i\leq n}(s_i)
				\end{align*}
			Then we have
				\begin{align*}
					mb_1
						& \leq a_1b_1+\ldots+a_nb_n\leq  Mb_1
				\end{align*}
		\end{theorem}
	
		\begin{proof}
			Write the sum $a_1b_1+\ldots+a_nb_n$ as the following.
				\begin{align*}
					a_1b_1+\ldots+a_nb_n
						& = s_1b_1+(s_2-s_1)b_2+\ldots+(s_n-s_{n-1})b_n\\
						& = s_1(b_1-b_2)+s_2(b_2-b_3)+\ldots+s_{n-1}(b_{n-1}-b_n)+s_nb_n
				\end{align*}
			Using $m\leq s_i\leq M$,
				\begin{align*}
					m(b_1-b_2)
						& \leq s_1(b_1-b_2) \leq M(b_1-b_2)\\
						& \vdots\\
					m(b_{n-1}-b_{n})
						& \leq s_{n-1}(b_{n-1}-b_n)\leq M(b_{n-1}-b_n)
				\end{align*}
			We additionally have $mb_n\leq s_nb_n\leq Mb_n$. Summing these inequalities together,
				\begin{align*}
					m(b_1-b_2+b_2-b_3+\ldots+b_{n-1}-b_n)+mb_n
						& \leq s_1(b_1-b_2)+s_2(b_2-b_3)+\ldots+s_{n-1}(b_{n-1}-b_n)+s_nb_n\\
						& \leq M(b_1-b_2+b_2-b_3+\ldots+b_{n-1}-b_n)+Mb_n\\
					mb_1
						& \leq a_1b_1+\ldots+a_nb_n\leq Mb_1
				\end{align*}
			This proves the inequality.
		\end{proof}
	% What about Aczel's inequality?
	\section{Practice Section}
	Try the following problems as warm up exercises.
		\begin{problem}
			For real numbers $x,y$, prove that
				\begin{align*}
					|x+y|
						& \leq|x|+|y|\\
					||x|-|y||
						& \leq|x+y|\\
					(|x|-|y|)^2
						& \leq|x^2-y^2|
				\end{align*}
			When does equality occur in the last inequality?
		\end{problem}
	The first inequality in this problem is also known as the \textit{triangle inequality}.
		\begin{problem}
			Prove Jordan's inequality
				\begin{align*}
					\dfrac{2}{\pi}
						& \leq\dfrac{\sin{\theta}}{\theta}<1
				\end{align*}
			for $0<|\theta|<\frac{\pi}{2}$.
		\end{problem}
	
		\begin{problem}
			Prove Redheffer's inequality.
				\begin{align*}
					\dfrac{\sin{\theta}}{\theta}
						& \geq\dfrac{\pi^2-\theta^2}{\pi^2+\theta^2}
				\end{align*}
		\end{problem}
	
		\begin{problem}[IMO $1960$]
			For which real value of $x$ does the inequality
				\begin{align*}
					\dfrac{4x^2}{(1-\sqrt{1+2x})^2}
						& <2x+9
				\end{align*}
			hold?
		\end{problem}
	The two inequalities above are not derived from each other. Also, you may have to put a bit of extra effort to prove them. Use calculus if that gives a faster solution.
\end{document}