\documentclass{subfile}

\begin{document}
	\section{Introduction}\label{sec:intro}% power mean, cauchy schwarz
	Let us start with the most fundamental inequality.
		\begin{align}
			x^2
				& \geq0\label{eqn:mother}
		\end{align}
	The first author calls it the mother of all inequality. Equality occurs if and only if $x=0$. We can extend it for $n$ variables.
		\begin{align}
			x_1^2+\ldots+x_n^2
				& \geq0\label{eqn:extendedmother}
		\end{align}
	Equality occurs if $x_i=0$ for all $1\leq i\leq n$. We can immediately get some results substituting $x$ with appropriate expressions. Substituting $x$ with $a-b$, we get
		\begin{align*}
			(a-b)^2
				& \geq0\\
			a^2+b^2
				& \geq2ab
		\end{align*}
	This is true for any real numbers $a,b$ and equality occurs if $a=b$. If $a,b$ are positive, then replacing $a$ and $b$ by $\sqrt{a}$ and $\sqrt{b}$ respectively, we get
		\begin{align*}
			a+b
				& \geq2\sqrt{ab}\\
			\iff\sqrt{\dfrac{a}{b}}+\sqrt{\dfrac{b}{a}}
				& \geq2\\
			\iff x+\dfrac{1}{x}
				& \geq2
		\end{align*}
	where $x=\dfrac{a}{b}$.
	
	Note that we can also write the first inequality as
		\begin{align*}
			\dfrac{a+b}{2}
				& \geq\sqrt{ab}
		\end{align*}
	Now, $\frac{a+b}{2}$ is the \textit{arithmetic mean} of $a$ and $b$. On the right side, $\sqrt{ab}$ is the \textit{geometric mean} of $a$ and $b$. So the inequality states that the arithmetic mean of two positive real numbers is greater than or equal to their geometric mean. We can also rewrite it as the following.
		\begin{align*}
			\sqrt{ab}\left(\dfrac{a+b}{2}\right)
				& \geq ab\\
			\sqrt{ab}
				& \geq\dfrac{2ab}{a+b}\\
			\sqrt{ab}
				& \geq\dfrac{2}{\frac{1}{a}+\frac{1}{b}}
		\end{align*}
	$\dfrac{2ab}{a+b}$ is the \textit{harmonic mean} of $a$ and $b$. S, this form of the inequality states that the geometric mean is larger than the harmonic mean. In fact, this result can be extensively generalized. First, we can consider $n$ variables $a_1,\ldots,a_n$ instead of just $a$ and $b$. Second, we can generalize the fact that $AM\geq GM\geq HM$. We will talk about this generalization in $\S$ \ref{sec:powermean}.
\end{document}