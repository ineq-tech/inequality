\documentclass[inequalities.tex]{subfile}

\begin{document}
	
	\section[Holder and Minkowski]{H\"{o}lder, Minkowski's Inequality}\label{sec:holdmink}
	
		\begin{theorem}[Generalized H\"{o}lder's Inequality]\label{thm:holder}
			Let $\mathbf{x}_{1},\ldots,\mathbf{x}_{n}$ be vectors of $n$ positive real numbers and $\omega$ be a weight vector. Then
				\begin{align*}
					\mathfrak{G}\left(\sum_{i=1}^{n}\mathbf{x}_{i},\omega\right)
					& \geq \sum_{i=1}^{n}\mathfrak{G}(\mathbf{x}_{i},\omega)
				\end{align*}
			Equality occurs if any two of the vectors are proportional.
		\end{theorem}
	This is not exactly the \index{H\"{o}lder's inequality}H\"{o}lder's inequality to be precise. The inequality appears in \textcite[Page $117$]{minkowski_1968}. To be strict, \textcite{holder_1889} proved the following theorem for  which we need to define conjugates. Two real numbers $u$ and $v$ are \textit{conjugate} if
		\begin{align*}
			\dfrac{1}{u}+\dfrac{1}{v}
				& = 1
		\end{align*}
	
		\begin{theorem}[H\"{o}lder's inequality]
			Let $u>1$ be a real number and $v$ be the conjugate of $u$. If $\alpha,\beta$ are complex numbers,
				\begin{align*}
					\langle\alpha,\beta\rangle
						& \leq \|\alpha\|_{u}\cdot\|\beta\|_{v}
				\end{align*}
			Equality occurs if $\alpha^{u}$ and $\beta^{v}$ are proportional.
		\end{theorem}
	H\"{o}lder's inequality can be proved with the help of \index{Young's inequality}\textcite{young_1912}.
		\begin{theorem}[Young's inequality]
			Let $a,b$ be positive real numbers and $u,v$ be conjugates. Then
				\begin{align*}
					ab
						& \leq \dfrac{a^{u}}{u}+\dfrac{b^{v}}{v}
				\end{align*}
		\end{theorem}
	The next result is usually known as \index{converse of H\"{o}lder's inequality}\textit{the converse of H\"{o}lder's inequality}.
		\begin{theorem}[The converse of H\"{o}lder's inequality]
			Let $\mathbf{a}$ and $\mathbf{b}$ be vectors of positive real numbers, $u$ and $v$ be conjugates.  If $B$ is a positive real number, then a necessary and sufficient condition that
				\begin{align*}
					\sum_{i=1}^{n}a_{i}^{u}
						& \leq A
				\end{align*}
			is that
				\begin{align*}
					\langle\mathbf{a},\mathbf{b}\rangle
						& \leq \sqrt[u]{A}\sqrt[v]{B}
				\end{align*}
			holds for all positive real numbers $b_{1},\ldots,b_{n}$ such that
				\begin{align*}
					\sum_{i=1}^{n}b_{i}^{v}
						& \leq B
				\end{align*}
		\end{theorem}
	\index{Ljapunov's inequality}\textcite{Ljapu_1901} proves the next theorem.
		\begin{theorem}[Ljapunov's inequality]
			Let $\mathbf{x},\mathbf{y}$ be two vectors of $n$ positive real numbers and $\omega$ be a weight vector. If $r,s,t$ are positive real numbers such that $r>s>t$,
				\begin{align*}
					\langle\mathbf{x},\mathbf{y}^{s}\rangle^{r-t}
						& \leq \langle\mathbf{x},\mathbf{y}^{t}\rangle^{r-s}\cdot\langle\mathbf{x},\mathbf{y}^{r}\rangle^{s-t}
				\end{align*}
		\end{theorem}
	\index{Minkowski's inequality}\textcite[Page $115-117$]{minkowski_1968} proves the following result.
		\begin{theorem}[Minkowski's Inequality]\label{thm:mink}
			Let $\mathbf{x}_{1},\ldots,\mathbf{x}_{n}$ be vectors of $n$ positive real numbers, $\omega$ be a weight vector and $r\neq1$ be a real number. Then
				\begin{align*}
					\mathfrak{M}_{r}\left(\sum_{i=1}^{n}\mathbf{x}_{i},\omega\right)
						&
							\begin{cases}
								\geq \sum_{i=1}^{n}\mathfrak{M}_{r}(\mathbf{x}_{i},\omega)& \mbox{if }r<1\\
								\leq \sum_{i=1}^{n}\mathfrak{M}_{r}(\mathbf{x}_{i}, \omega)& \mbox{otherwise}
							\end{cases}
				\end{align*}
		\end{theorem}
\end{document}