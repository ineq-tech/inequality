\documentclass{subfile}

\begin{document}
	\section{The UVW/PQR/ABC Method}\label{sec:uvw}
	The UVW method was popularized by \textcite{rozenberg_2011}. It has also been known as the ABC method or the PQR method although it is unclear what exactly the origin of this method is. Although some people believe it was originated at Vietnam where it was called the ABC (Abstract concreteness) method. Currently, \textcite{knudsen} is the most popular version of this technique but at the core, they all use the same idea. We will try to explain this method as clearly as possible with examples.

	Consider an inequality in three variables $a,b,c\in\mathbb{R}$. If the expression is symmetric on $a,b,c$, then the initial idea of PQR method was to write
		\begin{align*}
			a+b+c
				& = p\\
			ab+bc+ca
				& = q\\
			abc
				& = r
		\end{align*}
	so that $a,b,c$ are the roots of the equation
		\begin{align*}
			x^{3}-px^{2}+qx-r
				& = 0
		\end{align*}
	However, nowadays the most popular way to convert these is to write
		\begin{align*}
			a+b+c
				& = 3u\\
			ab+bc+ca
				& = 3v^{2}\\
			abc
				& = w^{3}
		\end{align*}
	Hence, the name UVW. While working with such transformations, be careful not to assume $v^{2}\geq 0$ by default. We have to consider the case where $v^{2}$ is negative as well. We do not know for sure that $a,b,c$ are all positive unless it is stated specifically. However, the following result is very nice when they are indeed positive. We will use the notations as stated above throughout this section and the names of the following theorems follow \textcite{knudsen}.
		\begin{theorem}[The Idiot Theorem]
			If $a,b,c\geq 0$, then $u\geq v\geq w$.
		\end{theorem}

		\begin{proof}
			Using $a^{2}+b^{2}+c^{2}\geq ab+bc+ca$ and $a+b+c\geq 3\sqrt[3]{abc}$, we immediately see that
				\begin{align*}
					(a+b+c)^{2}
						& \geq 3(ab+bc+ca)
				\end{align*}
			which implies $9u^{2}\geq 9v^{2}$ or $u^{2}\geq v^{2}$ and $u\geq w$. For proving $v\geq w$, we can use arithmetic-harmonic mean inequality
				\begin{align*}
					\dfrac{3}{\dfrac{1}{a}+\dfrac{1}{b}+\dfrac{1}{c}}
						& \leq \sqrt[3]{abc}\\
					\iff \dfrac{3abc}{ab+bc+ca}
						& \leq \sqrt[3]{abc}\\
					\iff ab+bc+ca
						& \geq 3\sqrt[3]{(abc)^{2}}\\
					\iff 3v^{2}
						& \geq 3w^{2}
				\end{align*}
			This proves the theorem.
		\end{proof}


		\begin{theorem}[The UVW theorem]\label{thm:uvw}
			Let $u,v,w$ be numbers such that $u,v^{2},w^{3}$ are real numbers. Then there exists real numbers $a,b,c$ such that
				\begin{align*}
					a+b+c
						& = 3u\\
					ab+bc+ca
						& = 3v^{2}\\
					abc
						& = w^{3}
				\end{align*}
			if and only if $u^{2}\geq v^{2}$ and
				\begin{align*}
					3uv^{2}-2u^{3}-2\sqrt{(u^{2}-v^{2})^{3}}
						& \leq w^{3} \leq 3uv^{2}-2u^{3}+2\sqrt{(u^{2}-v^{2})^{3}}
				\end{align*}
		\end{theorem}
	For proving this theorem, we will need the following result first.
		\begin{theorem}\label{thm:abc}
			$a,b,c$ are real numbers if and only if $(a-b)(b-c)(c-a)$ is a real number.
		\end{theorem}

		\begin{proof}
			The if part is obvious. So we will focus on the only if part. If $(a-b)(b-c)(c-a)$ is not real, then $a,b,c$ are not real either. Clearly, $a,b,c$ are the roots of $x^{3}-3ux^{2}+3vx-w^{3}=0$ by Vieta's formulas. If one of the roots is complex, say $a$, then another of $b,c$ is complex, say $b$. Moreover, we must have $a=z$ and $b=\bar{z}$ for some complex number $z$ since complex roots can appear only in conjugates. Then
				\begin{align*}
					(a-b)(b-c)(c-a)
						& = (z-\bar{z})(\bar{z}-c)(c-z)
				\end{align*}
			Letting $z=u+iv$ for some $u,v\in\mathbb{R}$, we have $\bar{z}=u-iv$ and
				\begin{align*}
					(a-b)(b-c)(c-a)
						& = -2iv(u-c+iv)(u-c-iv)\\
						& = -2iv\left((u-c)^{2}+v^{2}\right)
				\end{align*}
			This is obviously a complex number, so the claim is true unless $v=0$ which cannot hold since $z\not\in\mathbb{R}$.
		\end{proof}

		\begin{proof}[Proof of \nameref{thm:uvw}]
			For any $x\in\mathbb{R}$, we have $x^{2}\geq 0$. By \eqref{thm:abc}, $a,b,c\in\mathbb{R}$ implies that $(a-b)(b-c)(c-a)\in\mathbb{R}$ and so $(a-b)^{2}(b-c)^{2}(c-a)^{2}\in\mathbb{R}$. Then
				\begin{align*}
					(a-b)^{2}(b-c)^{2}(c-a)^{2}
						& \geq 0\\
					\iff 27\left(-(w^{3}-(3uv^{2}-2u^{3}))^{2}+4(u^{2}-v^{2})^{3}\right)
						& \geq 0\\
					\iff 4(u^{2}-v^{2})^{3}
						& \geq (w^{3}-(3uv^{2}-2u^{3}))^{2}\\
					\iff 2\sqrt{(u^{2}-v^{2})^{3}}
						& \geq \left|w^{3}-(3uv^{2}-2u^{3})\right|
				\end{align*}
			This proves the theorem.
		\end{proof}

		\begin{theorem}[The positivity theorem]\label{thm:positivity}
			$a,b,c$ are non-negative real numbers if and only if $u,v^{2},w^{3}$ are non-negative real numbers.
		\end{theorem}

		\begin{proof}
			The if part is obvious, so we prove the only if part. So we should prove that if any of $a,b,c$ are negative, then at lest one of $u,v^{2},w^{3}$ is negative. It is easy to see that $w^{3}=abc$ is negative if one or three of $a,b,c$ are negative. So, we are left with the case where two of $a,b,c$ are negative. Without loss of generality, assume that $a$ and $b$ are negative whereas $c$ is non-negative. Let $a=-x,b=-y$ so that
				\begin{align*}
					a+b+c
						& = 3u\\
					c-x-y
						& = 3u\\
					ab+bc+ca
						& = 3v^{2}\\
						& = xy-c(x+y)
				\end{align*}
			From this we can show that at least one of $u$ or $v$ has to be negative. Otherwise, if both $u$ and $v$ are non-negative, then $c-x-y\geq 0$ and $xy-c(x+y)\geq 0$. But this leads to a contradiction since
				\begin{align*}
					xy
						& \geq c(x+y)\\
						& \geq (x+y)^{2}\\
						& \geq 2xy
				\end{align*}
			by \nameref{thm:amgm}.
		\end{proof}

		\begin{theorem}[Tej's theorem]\label{thm:tej}
			Let $a,b,c$ be non-negative real numbers. Then we have the following:
				\begin{enumerate}
					\item If at least one value of $w^{3}$ exists for fixed $u$ and $v^{2}$ corresponding to $a,b,c$, then $w^{3}$ has a global maximum and minimum (see \gls{gls:globalextreme}). This maximum value is achieved when at least two of $a,b,c$ are equal. The minimum value is achieved when two of $a,b,c$ are equal or when one of them is $0$.
					\item If at least one value of $v^{2}$ exists for fixed $u$ and $w^{3}$ corresponding to $a,b,c$, then $v^{2}$ has a global maximum and minimum. This maximum value is achieved when at least two of $a,b,c$ are equal. The minimum value is achieved when two of $a,b,c$ are equal or when one of them is $0$.
					\item If at least one value of $u$ exists for fixed $v^{2}$ and $w^{3}$ corresponding to $a,b,c$, then $u$ has a global maximum and minimum. This maximum value is achieved when at least two of $a,b,c$ are equal. The minimum value is achieved when two of $a,b,c$ are equal or when one of them is $0$.
				\end{enumerate}
		\end{theorem}

		\begin{theorem}
			Every symmetric inequality of degree at most $5$ has to be proved only for $a_{1}=a_{2}$ or $a_{n}=0$.
		\end{theorem}

		\begin{proof}
			Using \gls{gls:elempoly},
		\end{proof}
\end{document}