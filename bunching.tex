\documentclass[inequalities.tex]{subfile}

\begin{document}
	\section[Karamata]{Karamata's Inequality}\label{sec:karamata}
	\index{Karamata's inequality}\textcite{karamata_1932} proves the next theorem regarding convex functions when one vector majorizes another.
		\begin{theorem}[Karamata's inequality]
			Let $f$ be a convex function and $\mathbf{a}$ and $\mathbf{b}$ be two vectors such that $\mathbf{a}\succ\mathbf{b}$. Then
				\begin{align*}
					\sum_{i=1}^{n}f(a_{i})
						& \geq \sum_{i=1}^{n}f(b_{i})
				\end{align*}
		\end{theorem}
	\section[Bunching: Muirhead]{The Bunching Method: Murihead's Inequality}\label{sec:bunching}
	
	\index{Muirhead's inequality}\textcite{muirhead_1902} proves a very important result regarding symmetric sums and means. It is widely used for solving problems, specially coupled with some other method such as \hyperref[ch:buffalo]{The Buffalo Way}. It is also known as \index{the bunching principle}\textit{the bunching principle} among Americans.
		\begin{theorem}[Muirhead's inequality]
			Let $\mathbf{x}$ be a vector of positive real numbers and $\mathbf{a}$ be dominated by $\mathbf{b}$. Then
				\begin{align*}
					\mathfrak{M}[\mathbf{a}](\mathbf{x})
						& \leq \mathfrak{M}[\mathbf{b}](\mathbf{x})
				\end{align*}
		\end{theorem}
	\textcite{paris_vencovska_2009} proves a generalization of this inequality but the result is of little practical use for Olympiad purposes.
		\begin{theorem}[Generalization of Muirhead's inequality]
			Let $\mathbf{x}$ and $\mathbf{y}$ be two vectors of positive real numbers such that $\mathbf{x}\succ\mathbf{y}$. If $p_{1},\ldots,p_{k}$ are non-negative real numbers, then
				\begin{align*}
					\sum_{\substack{S_{1}\cup\ldots\cup S_{r}=\{1,\ldots,k\}\\S_{i}\cap S_{j}=\phi}}\prod_{j=1}^{r}\left(\sum_{i\in S_{j}}p_{i}\right)^{\square x_{j}}
						& \geq \sum_{\substack{S_{1}\cup\ldots\cup S_{r}=\{1,\ldots,k\}\\S_{i}\cap S_{j}=\phi}}\prod_{j=1}^{r}\left(\sum_{i\in S_{j}}p_{i}\right)^{\square y_{j}}
				\end{align*}
			where $\square$ in $\square x_{j}$ denotes that in the expansion of this power, we only consider the terms with non-zero power of $p_{i}$.
		\end{theorem}
\end{document}